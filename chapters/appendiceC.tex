Le derivate delle funzioni di Bessel possono essere espresse attraverso altre funzioni di Bessel(vedi 9.1.30
Abramowitz-Stegun\cite{AbramSteg}).
Da questa formula si trova la relazione seguente
\[
 \frac{1}{z} \frac{d}{dz} z J_1(z) = J_0(z)
\]
Dalla (\ref{eq:scattering-amplitude-bessel-zero}) abbiamo
\begin{align*}
    f(\bm{q}) & = ik \int_{0}^R rJ_{0}(qr) \, dr \\
    &= i \frac{k}{q^{2}}\int_{0}^{qR}zJ_{0}(z) \, dz \\
    & = i \frac{k}{q^{2}}\int_{0}^{qR}z \frac{1}{z} \frac{d}{dz}zJ_{1}(z) \, dz \\
    & = i \frac{k}{q^{2}} zJ_{1}(z) \big{|}_{0}^{qR}
\end{align*}
da cui infine
\[
f(\bm{q}) = i \frac{kR}{q}J_{1}(qR)
\]