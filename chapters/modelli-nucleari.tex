Esamineremo ora alcuni modelli che sono stati costruiti per finalizzare la descrizione del nucleo.

Una classificazione di alcuni modelli nucleari in base al tipo di interazione tra nucleoni è la seguente:
\begin{multicols}{2}
	\begin{center}
		Models with strong interaction between nucleons:
	\end{center}
	\begin{itemize}
		\item \textbf{Liquid drop model} (Modello a goccia);
		\item \textbf{Shell model} (Modello a shell);
		\item $ \alpha-$particle model.
	\end{itemize}
   \begin{center}
	   Nucleons interact with the nearest neighbors and practically don‘t move: mean free path \[ \lambda \ll R\]
   \end{center}
	\columnbreak

	\begin{center}
		Models of non-interacting nucleons :
	\end{center}
	\begin{itemize}
		\item \textbf{Fermi gas model} (Modello a gas di fermioni);
		\item Optical model (Modello ottico).
	\end{itemize}
	\begin{center}
		Nucleons move freely inside the nucleus: mean free path \[ \lambda \sim R\]
	\end{center}
\end{multicols}
\begin{center}
($ R$ being the nuclear radius)
\end{center}
\section{Il modello nucleare a goccia}\label{sec:il-modello-nucleare-a-goccia}
Il grafico della energia di legame media per nucleone in Figura~\ref{fig:en-legame-graph} descrive fondamentali proprietà dei nuclei.
Non deve quindi sorprendere che sin dagli esordi della fisica nucleare si sia tentato di dedurlo per via teorica.
I primi tentativi risalgono al 1935 quando Bohr ne descrisse gli aspetti essenziali nel contesto del \textbf{modello a goccia del nucleo},
basato sul corto raggio della interazione nucleare e su alcune analogie tra il comportamento dei nucleoni nel nucleo e le porzioni di fluido in un liquido.
Nei decenni seguenti il modello è stato via via perfezionato fino a giungere alla versione che discuteremo nel prosieguo.
%Il grafico (Fig.~\ref{fig:en-legame-graph}) della energia di legame media per nucleone commentato in precedenza
%contiene un certo numero di importanti indicazioni
%sulle proprietà della forza nucleare che sono alla base di un primo
%modello del nucleo - suggerito da Bohr nel 1935 - fondato essenzialmente
%sul \textbf{raggio finito} della interazione nucleare.
%Similmente alle forze intramolecolari a corto raggio nei liquidi,
%tale fatto determina una certa analogia tra il comportamento dei nucleoni
%nel nucleo e le diverse porzioni di un fluido meccanico incomprimibile, ragione che
%giustifica il nome spesso usato di \textbf{modello a goccia}.
\begin{figure}
	\centering
	\includegraphics{figs/liquid-drop-model}
	\caption{Rappresentazione grafica del significato fisico dei termini presenti nel nuclear liquid drop model.}
	\label{fig:liquid-drop-model}
\end{figure}

Il suddetto modello ha base \textbf{fenomenologica}; la sola ipotesi
dell'andamento a corto raggio della forza si rivelerà, tuttavia, non sufficiente a
giustificare l'andamento fisico reale.

Come accennato, l'energia di legame tende ad assumere rapidamente il
valore medio di circa \(8MeV\) per nucleone (saturazione) il che indica
una energia di legame del nucleo proporzionale al numero di nucleoni
\begin{equation}
	\frac{B}{A} \simeq 8 MeV \qquad B \simeq 8 MeV \times A
    \label{eq:saturation-binding-energy-per-nucleon}
\end{equation}
Ora, se la forza nucleare si comportasse come una forza a lungo
raggio (come le forze gravitazionali o elettromagnetiche) ogni nucleone
interagirebbe con tutti i rimanenti altri per cui dovremmo attenderci
una energia di legame del nucleo tendenzialmente \emph{proporzionale al numero
di coppie} di nucleoni e dunque quadratica in A
\[
	B \propto \frac{A(A-1)}{2} \propto A^{2} \qquad
\]
Poiché i dati sulla energia di legame escludono questo tipo di
comportamento, dobbiamo concludere che ogni nucleone del nucleo
interagisce solo con un numero di fisso di nucleoni vicini per cui
concludiamo che \textbf{l'interazione forte ha un raggio d'azione finito}
dell'ordine di grandezza delle dimensioni del nucleone stesso.

Tale conclusione è in perfetto accordo con i dati sulla sezione d'urto
di neutroni su nuclei analizzati in precedenza (vedi sezione~\ref{sec:raggio-nucleare}) che indicavano un \textbf{volume
nucleare proporzionale al numero di nucleoni}, ovvero una densità
volumetrica di nucleoni uniforme sul volume nucleare, fatto spiegabile
solo postulando la esistenza di una forza d'interazione tra nucleoni a
corto raggio.

Il primo tentativo di superare questo limite consiste nell'introdurre un
termine, sulla base della (\ref{eq:saturation-binding-energy-per-nucleon})
\begin{equation}
	B = a_{v} A
   \label{eq:volume-term-drop-model}
\end{equation}
dove la costante \(a_{v}\) viene detta \textbf{termine di volume}.
\begin{marginfigure}
	\includegraphics{figs/goccia1}
	%    \caption{This is a margin figure.}
	\label{fig:goccia1}
\end{marginfigure}
Con un tale andamento \(B / A\) , però, si finisce per ottenere una sovrastima
del volume.
La deviazione più rilevante si manifesta per valori piccoli di A dove
l'energia media di legame è molto inferiore a quanto previsto dalla
formula.
\bigskip

Si può allora osservare che, assumendo la forza nucleare a
corto raggio, si deve tenere conto che un nucleone prossimo alla
superficie del nucleo interagirà con un numero di nucleoni inferiore a
quello con cui interagirebbe qualora si trovasse all'interno del nucleo
stesso.
Ciò comporta che i nucleoni superficiali contribuiranno in
misura minore alla energia di legame nucleare di quelli interni al
volume.
Assumendo il nucleo di \textbf{forma sferica}, il numero di
nucleoni prossimi alla superficie sarà proporzionale a \(R^{2}\).
Dalla trattazione precedente (eq. \ref{eq:nuclear-radius-skin}) sappiamo che (omettendo il termine di `skin'
nucleare) \begin{gather*}
	R_{\text{nuc}} = r_{0}A^{1/3}\\
	4 \pi R^{2} = 4 \pi (r_{0}A^{1/3})^2 \implies R_{\text{nuc}} \propto A^{2/3}
\end{gather*} per cui vi deve essere un termine che deve provocare un difetto di
energia di legame proporzionale ad \(A^{2/3}\): \[
	B = a_{v}A - a_{s}A^{2/3}
\] dove la costante \(a_{s}\) viene detta \textbf{termine di
	superficie}.
L'andamento di \(B / A\) con questa ulteriore correzione
può apprezzarsi a lato.
\begin{marginfigure}
	\includegraphics{figs/goccia2}
	%    \caption{This is a margin figure.}
	\label{fig:goccia2}
\end{marginfigure}
\bigskip

Un ulteriore miglioramento può essere ottenuto tenendo presente che i
protoni del nucleo si \emph{respingono elettrostaticamente} diminuendo
quindi il lavoro necessario per separarli dal nucleo stesso.
In effetti
se, per assurdo, si avesse un nucleo composto solo di protoni(senza
neutroni) il lavoro da spendere per mantenerne la configurazione sarebbe
sicuramente maggiore.

Ipotizzando una \emph{distribuzione di protoni uniforme} nel volume
nucleare (ricordiamo essere sferico dall'ipotesi precedente), otteniamo
la seguente espressione del lavoro fatto dalle forze coulombiane
repulsive per separare la carica nucleare
\begin{gather*}
	\delta L = \int _{R}^{\infty} \left( \frac{q \delta q}{4 \pi \epsilon_{0}r^{2}}  \hat{\bm{r}} \right)(dr \hat{\bm{r}})=
	- \frac{q \delta q}{4 \pi \epsilon_{0}r} \bigg |_{R}^{\infty} =
	- \frac{q \delta q}{4 \pi \epsilon_{0}R}\\
	q = \rho \frac{ 4}{3} \pi R^{3} \qquad \delta q = \rho 4 \pi R^{2} dR\\
	\delta L = \frac{1}{4 \pi \epsilon_{0}R}\left( \rho \frac{ 4}{3}\pi R^{3} \right)(\rho 4 \pi R^{2} dR) = \frac{4 \pi \rho^{2}}{3 \epsilon_{0}}R^{4}dR\\
	L = \int \delta l = \frac{4 \pi \rho^{2}}{15 \epsilon_{0}}R_{0}^{5} \qquad Q = \rho \frac{ 4}{3} \pi R_{0}^{3} = Ze\\
	L = \frac{3}{20 \pi \epsilon_{0}} \frac{Q^{2}}{R_{0}} = \frac{3e^{2}}{20 \pi \epsilon_{0}r_{0}} \frac{Z^{2}}{A^{1/3}}
	\implies L \propto \frac{ Z^{2}}{A^{1/3}}
\end{gather*} Ne consegue che la energia di legame nucleare dovrà essere corretta
sottraendo un termine proporzionale a \(Z^{2} / A^{1/3}\) per cui
l'espressione dell'energia di legame acquisirà la forma seguente
\begin{equation}
	B = a_{v}A - a_{s}A^{2/3} - a_{c} \frac{Z^{2}}{A^{1/3}}
	\label{eq:coulomb-term-drop-model}
\end{equation}
 dove la nuova costante \(a_{c}\) viene detta \textbf{termine
	coulombiano}.
Il termine coulombiano deve chiaramente annullarsi per
\(A=1\) (non c'e interazione elettrostatica) per cui l'unica forma
possibile è
\begin{equation}
	L \propto \frac{Z(Z-1)}{A^{1/3}} \quad
   \label{eq:work-to-seperate-uniformly-distributed-protons}
\end{equation}
\begin{marginfigure}
	\includegraphics{figs/goccia3}
	%    \caption{This is a margin figure.}
	\label{fig:goccia3}
\end{marginfigure}

L'andamento $B / A$ risulta ulteriormente migliorato (si tenga presente
che nei nuclei stabili si ha approssimativamente \(Z=A/2\) per cui il
termine coulombiano sottrae un contributo crescente con \(A^{5/3}\)).

Se il modellino fenomenologico costruito finora fosse completo dovremmo
concludere che i nuclei più stabili (\(B= B_{max}\)) sono quelli con
\(Z=0\) ovvero i nuclei di soli neutroni.
Tale fatto è palesemente
contraddetto dai dati sperimentali i quali mostrano che i nuclei stabili
hanno un numero di protoni di poco inferiore a quello dei neutroni (la
differenza tra neutroni e protoni tende a crescere con il numero
atomico).
Il nostro modello -- basato sulla natura a corto raggio della
interazione nucleare -- non offre alcun appiglio per dare un fondamento
fisico a questo stato di cose che potrà essere spiegato solo nel
contesto della meccanica quantistica attraverso il \emph{principio di
	esclusione di Pauli}.
In questa situazione l'unica possibilità è quella
di introdurre un termine `ad hoc' capace di descrivere i dati
sperimentali.

Bisogna quindi fare in modo che il modello ci dica che i nuclei piu
stabili sono quelli con lo stesso numero di protoni e neutroni.
Raggiungiamo l'obiettivo introducendo un nuovo termine del tipo:
\[
	Z \simeq \frac{A}{2} \quad A - 2Z \simeq 0
\]
Ricordando però che con il crescere di \(A, Z\) tende ad essere via
via più piccolo di \(A/2\) (il quoziente protoni/neutroni diminuisce con
\(A\)), tale termine correttivo dovrà seguire una legge inversa ad A
modulata da un qualche esponente.
I dati indicano che la prima potenza è
sufficiente per cui abbiamo la seguente espressione della energia di
legame nucleare
\[
	B = a_{v}A - a_{s}A^{2/3} - a_{c} \frac{Z(Z-1)}{A^{1/3}} - \frac{a_{a}(A-2Z)^{2}}{A}
\]
dove la nuova costante \(a_{a}\) viene detta \textbf{termine di asimmetria}.
Notiamo che la presenza di \(A\) a denominatore è
giustificata osservando l'andamento dei dati sperimentali nel grafico~\ref{fig:en-legame-graph}: tanto piu \(A\) è grande tanto piu \(B / A\) devia
dalla tangente alla curva (quindi verso il basso).
\bigskip

Per completare il modello è necessario tenere conto di un ulteriore
proprietà dei nuclei.
I dati sperimentali mostrano che tra i 254 nuclei
stabili noti ben 148 sono del tipo \textbf{pari-pari} (un numero pari
sia di protoni che di neutroni), 101 sono del tipo \textbf{pari-dispari}
(un numero pari di protoni ma dispari di neutroni o viceversa) e solo 5
sono del tipo dispari-dispari (riportati in tabella \ref{tab:odd-odd-stable-nuclei}).
\begin{table}
	\centering
	\begin{tabular}{|c|}
		\hline
		$\isotope[2][1]{\mathlarger{H}}$ \\ \hline
		$\isotope[6][3]{\mathlarger{Li}}$ \\ \hline
		$\isotope[10][5]{\mathlarger{B}}$ \\ \hline
		$\isotope[14][7]{\mathlarger{N}}$ \\ \hline
		$\isotope[180m][73]{\mathlarger{Ta}}$ \\ \hline
	\end{tabular}
	\caption{Tabella dei nuclei stabili con both $ A,Z$ dispari. L'$m$ ad apice indica la metastabilità del nuclide. \\
	Si veda \url{https://en.wikipedia.org/wiki/Metastability#Nuclear_physics} a tale proposito.}
	\label{tab:odd-odd-stable-nuclei}
\end{table}

Similmente, tra i 35 nuclei a lunga vita media si hanno 22 pari-pari, 9 pari-dispari e 4
dispari-dispari.

Tali dati sembrano suggerire che per qualche motivo la forza forte tra
nucleoni da luogo a nuclei di maggiore stabilità quando vengono legati
\textbf{numeri pari} di neutroni e protoni, un fatto che trova una sua
diretta evidenza nell'andamento della energia di legame per nucleone
della serie isotopica dello xenon in Figura~\ref{fig:binding-energy-xenon}.
Tali dati sembrano suggerire che per qualche motivo la forza forte tra nucleoni da luogo a nuclei di maggiore stabilità
quando vengono legati \textbf{numeri pari} di neutroni e protoni, un fatto che trova una sua diretta evidenza nell’andamento della energia di legame per nucleone della serie isotopica dello xenon.
\begin{marginfigure}
	\includegraphics[scale = 1.5]{figs/goccia5}
	\caption{Andamento dell'energia di legame per la serie isotopica dello Xenon.}
	\label{fig:binding-energy-xenon}
\end{marginfigure}
Tenendo presente che il nucleo di Xenon ha \(Z=54\) protoni, si può
infatti constatare che gli isotopi con un numero pari di neutroni hanno
una maggiore energia di legame per nucleone.
Per rendere conto di questo
fatto si introduce un nuovo coefficiente detto di \textbf{pairing} \[
	a_{p} \frac{\delta}{A^{3/4}}
\] con \[
	\delta =
	\begin{cases}
		+1    \qquad  \text{pari-pari}       \\
		\ 0  \qquad  \ \ \text{pari-dispari} \\
		-1  \qquad  \text{dispari-dispari}
	\end{cases}
\] Giungiamo così alla seguente espressione complessiva della energia di
legame nucleare detta anche \textbf{formula semiempirica della energia
	di legame nucleare} o \textbf{formula di Weizsacker} della energia di
legame nucleare
\begin{equation}
	\boxed{    B = a_{v}A - a_{s}A^{2/3} - a_{c} \frac{Z(Z-1)}{A^{1/3}} - a_{a}\frac{(A-2Z)^{2}}{A} +     a_{p} \frac{\delta}{A^{3/4}}}
	\label{eq:weizsacker-formula}
\end{equation} Essa dipende da 5 parametri(vedi Fig.~\ref{fig:liquid-drop-model}) il cui valore numerico
viene determinato adattando la formula ai dati sperimentali di \(B/A\).

A titolo di esempio, una possibile combinazione di valori è la seguente (i
valori sono in \(MeV\)):
\begin{equation}
	a_{v} = 15.5 \quad a_{s} = 16.8 \quad a_{c} = 0.72 \quad a_{a} = 23.0 \quad a_{p} = 34.0
	\label{eq:typical-values-drop-model}
\end{equation}
La formula della energia di legame nucleare è utile - come strumento di
calcolo - perchè suggerisce una prima interpretazione del nucleo e delle
forze che lo tengono insieme (vedi esempi di applicazione a pagg. 74-75 delle dispense).

Da essa deduciamo che le forze tra nucleoni devono essere \textbf{molto
	intense} ma a \textbf{short range} (proprietà di saturazione), producono
una distribuzione spaziale tendenzialmente uniforme di nucleoni e
conducono ad una espressione della energia di legame con termini di
volume e superficie in analogia con quanto accade per i liquidi, ragione
che giustifica il nome spesso usato di \textbf{modello nucleare a
	goccia}.

Il \emph{modello a gas di fermioni} (giustificabile con considerazioni
quantomeccaniche) riuscirà a rendere conto del termine \(a_{a}\) mentre
fallisce per quello di pairing.
Vedremo, infine, che il \emph{modello a shell} sarà quello piu preciso.

%%%%%%%%%%%%%%%%%%%%%%%%%%%%%%%%%%%%%%%%%%%%%%%%%%%%%%%%%%%%%%%%%%%%%%%%%%%%%%%%%%%%%%%%%%%%%%%%%%%%%%%%%%%%%%%%%%%%%%%%
%%%%%%%%%%%%%%%%%%%%%%%%%%%%%%%%%%%%%%%%%%%%%%%%%%%%%%%%%%%%%%%%%%%%%%%%%%%%%%%%%%%%%%%%%%%%%%%%%%%%%%%%%%%%%%%%%%%%%%%%
%%%%%%%%%%%%%%%%%%%%%%%%%%%%%%%%%%%%%%%%%%%%%%%%%%%%%%%%%%%%%%%%%%%%%%%%%%%%%%%%%%%%%%%%%%%%%%%%%%%%%%%%%%%%%%%%%%%%%%%%
\section{Il nucleo come gas di fermioni}\label{sec:il-nucleo-come-gas-di-fermioni}



Il modello a goccia del nucleo - essenzialmente fondato sulla natura a corto raggio delle interazioni forti - riesce a descrivere alcune proprietà del nucleo tra le quali l’esistenza di una energia di legame con termini di volume, superficie e coulombiano.
Non riesce invece a rendere conto in nessun modo dei termini di asimmetria e accoppiamento, chiaramente richiesti dai dati sperimentali, ma che devono essere introdotti ‘a mano’ nella formula di Weizsacker.
Tale fatto dimostra che oltre alla natura a corto raggio delle interazioni forti, nel nucleo giocano un ruolo di rilievo altre proprietà trascurate dal modello a goccia, presumibilmente legate alla natura quantomeccanica dei suoi costituenti.
Il modello nucleare a gas di fermioni introduce nel gioco alcuni essenziali proprietà quantomeccaniche in una forma il più possibile semplificata.

\subsection{Aspetti generali}\label{sec:aspetti-generali}

Le considerazioni fisiche che conducono alla formulazione del modello a gas di fermioni possono essere riassunte nei seguenti punti:
\begin{itemize}
	\item l’interazione forte che lega protoni e neutroni nel nucleo è a corto raggio
	e dell’ordine del raggio dei nucleoni stessi;
	\item ogni nucleone sarà quindi soggetto alla forza forte dei nucleoni
	immediatamente a contatto che tenderanno così a disporsi con densità
	volumetrica approssimativamente costante;
	\item ciò comporta che la risultante delle forze agenti sul singolo nucleone sia
	mediamente nulla quando questo si trova all’interno al nucleo e mediamente non nulla e diretta verso il centro quando questo si trova sulla superficie del nucleo;
	\item ciò si riassume in una forza dipendente dalla posizione, nulla all’interno di un volume sferico di raggio $R$ (raggio nucleare) e non nulla e centrale sulla sua superficie;
	\item essendo $ \bm{F} = - grad \, U$, tale forza può essere descritta da un potenziale
	nullo all’interno del volume sferico di raggio R, che sale con una certa ripidità in corrispondenza della superficie sferica,
	fino a raggiungere un valore costante al di fuori di essa. \\
	In sintesi, una \textbf{buca sferica di potenziale di raggio R} all’interno della quale vanno a collocarsi sia i neutroni che i protoni.
\end{itemize}
\bisgkip

Dato questo assetto delle forze, la meccanica quantistica fa il resto.
Infatti sappiamo che
\begin{itemize}
	\item una particella microscopica vincolata a rimanere in una regione di spazio
	limitata (nel nostro caso all’interno della buca di potenziale) deve annullare la propria funzione d’onda più o meno sulla superficie di tale regione ed all’esterno di essa;
	\item ciò comporta la quantizzazione delle lunghezze d’onda di De Broglie dei neutroni e dei protoni e, con esse, della loro energia (fenomeno analogo alla quantizzazione delle lunghezze d’onda in una corda vibrante) cosicchè gli stati quantomeccanici vanno a costituire una serie discreta e numerabile;
	\item dato che protoni e neutroni hanno spin $s=\frac{1}{2}$, essi vanno a costituire due insiemi di fermioni identici soggetti alle restrizioni del principio di esclusione;
	\item organizzando la serie discreta degli stati quantomeccanici possibili secondo i valori crescenti delle loro energie, il principio di esclusione permette di collocare in ciascuno di essi un solo protone ed un solo neutrone;
	\item nello stato fondamentale di minima energia del nucleo, neutroni e protoni nucleari riempiranno allora dal basso tutti gli stati quantomeccanici fino ad un livello energetico massimo detto \textbf{livello di Fermi} (nello stato fondamentale di minima energia dunque, i neutroni ed i protoni non sono immobili ma possiedono una certa energia cinetica crescente).
\end{itemize}

Le precedenti considerazioni ci conducono a concludere che la \emph{natura della forza media nucleare}, combinata con la
\emph{natura quantomeccanica dei suoi costituenti}, \textbf{assimila il nucleo nel suo stato fondamentale ad un doppio
gas di neutroni e protoni in condizioni di massima degenerazione}.
Parliamo di `gas’ poiché neutroni e protoni – come si è detto – si muovono essenzialmente liberi all’interno del volume nucleare subendo le sole forze di contenimento delle pareti nucleari.
Il termine ‘doppio’ ricorda che le restrizioni del principio di esclusione si applicano separatamente ai protoni ed ai neutroni.
Infine, il termine ‘degenerazione’ si riferisce al fatto che gli stati quantomeccanici quantizzati si riempiono tutti a partire da quelli meno energetici fino al livello di Fermi.
Come noto, tale condizione si realizza nei \textbf{gas di fermioni identici alle bassissime temperature che trovano così una loro sorprendente relazione con il nucleo nel suo stato fondamentale}.
Da questo punto di vista la differenza è solo quantitativa poiché i gas quantomeccanici contano un numero di elementi dell’ordine del numero di Avogadro mentre il doppio gas nucleare risulta costituito da decine di unità un fatto che – come già osservato – rende sostanzialmente inapplicabili i metodi della meccanica statistica.

Nel prosieguo, l’approccio del modello a gas di fermioni verrà utilizzato per costruire una formula della energia di legame più evoluta di quella fornita dal modello a goccia nella speranza che la meccanica quantistica possa rendere conto dei termini introdotti in modo puramente fenomenologico nella formula di Weizsacker.

\subsection{La funzione d'onda del nucleone}\label{sec:funzione-onda-nucleone}

Il primo passo è quello di scrivere la funzione d’onda di $N$ neutroni e $Z$ protoni soggetti alla buca di potenziale sferica che descrive la forza media agente su di loro all’interno del nucleo.
Senza perdere in generalità, possiamo semplificare i calcoli immaginando una \emph{buca di potenziale cubica di lato L} all’interno della quale si trova un \emph{nucleone} (protone o neutrone) in uno stato quantomeccanico descritto da un’onda piana di De Broglie
\[
\psi(\bm{r},t) = A e^{ i(\bm{k} \cdot \bm{r} - \omega t) } =
Ae^{ \frac{i}{\hslash} (\bm{p} \cdot \bm{r} -Et)  } =
(Ae^{ \frac{i}{\hslash}p_{x}x }e^{ \frac{i}{\hslash}p_{y}y }e^{ \frac{i}{\hslash}p_{z}z })e^{ - \frac{i}{\hslash}Et }
\]
Con tutta evidenza, per descrivere correttamente lo stato quantomeccanico del nucleone, tale \emph{funzione d’onda dovrà annullarsi sulla superficie cubica}.
\begin{marginfigure}
	\includegraphics{figs/cube-fermion-gas1}
%	\caption{Fermi surface.}
%	\label{fig:goccia1}
\end{marginfigure}
Gli esponenziali complessi non possono annullarsi per cui è necessario ipotizzare che la funzione d’onda contenga solo le parti sinusoidali o cosinusoidali di tali esponenziali:
\begin{equation}
	\psi(\bm{r},t) = A \sin\left(  \frac{1}{\hslash}p_{x}x \right) \sin\left(  \frac{1}{\hslash}p_{y}y \right) \sin\left(  \frac{1}{\hslash}p_{z}z \right) e^{ - \frac{i}{\hslash}Et }
	\label{eq:wave-function-nucleon-fermion-gas-non-normalized}
\end{equation}
In questa forma è agevole imporre l’annullamento della funzione d’onda sulle pareti della cavità nucleare ovvero sui piani $x=0$, $x=L$;  $y=0$, $y=L$; $z=0$, $z=L$.
Ragionando ad esempio lungo la direzione $x$ abbiamo le condizioni
\[
\frac{1}{\hslash} p_{x}x \big|_{x = 0,L} = 0
\]
sempre soddisfatta in $x=0$ e soddisfatta in $x=L$ solo se
\[
\frac{1}{\hslash}p_{x}L = n_{x} \pi \qquad n_{x} = 1,2, \dots, N
\]
Dato che condizioni analoghe si trovano immediatamente anche nelle direzioni $y$ e $z$, giungiamo alla conclusione che
le componenti cartesiane della \emph{quantità di moto} del nucleone soddisfano le seguenti \emph{condizioni di quantizzazione}
\begin{equation}
	p_{x} = n_{x} \frac{\pi \hslash}{L} \quad
	p_{y} = n_{y} \frac{\pi \hslash}{L} \quad
	p_{z} = n_{z} \frac{\pi \hslash}{L} \qquad
	n_{x},n_{y},n_{z} = 1,2, \dots , n
	\label{eq:quantized-momentum-fermions-gas}
\end{equation}
Come anticipato, ciò comporta che pure \emph{l’energia} del nucleone soddisfi la seguente \emph{condizione di quantizzazione}\sidenote{
 Ricordiamo che l'energia considerata è quella cinetica in quanto nel modello che stiamo costruendo $ V(r) = 0$ per $ r< R$.
}
\begin{equation}
	E = \frac{p^{2}}{2 M} = \frac{1}{2 M} \left( \frac{\pi \hslash}{L} \right)^{2} (n_{x}^{2} + n_{y}^{2}+n_{z}^{2}) \qquad
	n_{x},n_{y},n_{z} = 1,2, \dots , n
	\label{eq:energy-nucleon-fermions-gas}
\end{equation}
Sostituendo le (\ref{eq:quantized-momentum-fermions-gas}) nella (\ref{eq:wave-function-nucleon-fermion-gas-non-normalized})
ed imponendo la condizione di normalizzazione sul volume nucleare
\begin{gather*}
    \iiint_{V} |\psi(\bm{r},t) |^{2} \, dV = 1 \iff
1 = A^{2} \prod_{i = 1}^{3} \int_{0}^{L} \sin ^{2}\left( n_{x_{i}} \frac{\pi x_{i}}{L} \right) \, dx_{i}\\
    1 = A^{2} \left( \frac{L}{2} \right)^{3} \Longrightarrow A = \sqrt{ \frac{8}{L^{3}} }
\end{gather*}
otteniamo la \textbf{funzione d’onda del nucleone nel volume nucleare cubico di lato L}
\begin{equation}
	\psi(\bm{r},t) = \sqrt{ \frac{8}{L^{3}}} \sin\left( n_{x} \frac{\pi x}{L} \right) \sin\left( n_{y} \frac{\pi y}{L} \right)
	\sin\left( n_{z} \frac{\pi z}{L} \right) e^{ -\frac{i}{\hslash}Et }
	\label{eq:wave-function-nucleon-fermion-gas-normalized}
\end{equation}
Si comprende allora che gli stati quantomeccanici del nucleone vanno a costituire una \emph{sequenza discreta e numerabile di stati univocamente identificati da una terna ordinata di numeri naturali (non nulli e positivi)} $n_x, n_y, n_z$  \emph{detti numeri quantici dello stato}.

\subsection{Nucleoni e principio di esclusione} \label{sec:nucleoni-principio-di-esclusione}
%
%Trovata l’espressione degli stati quantomeccanici di un generico nucleone nella cavità nucleare, vogliamo ora calcolare \emph{il numero di stati che possiedono una energia minore o uguale ad un certo prefissato valore}.
%Con tutta evidenza - dato il legame esistente tra energia e quantità di moto – ciò equivale a calcolare il numero di stati che possiedono un modulo della quantità di moto minore od uguale ad un certo prefissato valore.
%
%Premesso che il calcolo si base sulla stessa tecnica applicata a suo tempo al caso della radiazione elettromagnetica di cavità, si comincia \emph{introducendo uno spazio tridimensionale i cui punti rappresentano i vettori della quantità di moto che avranno così proiezioni} $p_x, p_y \ \text{e} \ p_z$.
%
%Tenendo conto delle relazioni di quantizzazione della quantità di moto (\ref{eq:quantized-momentum-fermions-gas}), si comprende immediatamente che – in tale spazio – i valori delle quantità di moto accessibili al nucleone si collocano nel primo ottante sulle intersezioni di un grigliato tridimensionale di passo $\frac{\pi \hslash}{L}$.
%A ciascuna di tali intersezioni corrisponde anche una determinata terna ordinata di numeri quantici $n_{x}, n_{y}$ e $n_{z}$ e, con essa, un determinato stato quantomeccanico.
%
Prima di proseguire sono necessarie alcune precisazioni.
A rigore, la funzione d’onda (\ref{eq:wave-function-nucleon-fermion-gas-normalized}) descrive lo stato di uno dei nucleoni del nucleo dato che soddisfa le condizioni imposte dal potenziale nucleare medio. Lo stato del nucleo nel suo complesso é allora descritto dal prodotto tensoriale di ‘A’ funzioni del tipo (\ref{eq:wave-function-nucleon-fermion-gas-normalized}), ciascuna con la propria terna $n_{x}, n_{y}$ e $n_{z}$ di \emph{numeri quantici di stato}. Ciò premesso, si capisce che non fa differenza descrivere lo stato del nucleo nel modo suddetto oppure immaginare di utilizzare una sola funzione d’onda del tipo (\ref{eq:wave-function-nucleon-fermion-gas-normalized}) specificando però per ciascuna terna $n_{x},n_{y},n_{z}$ il numero di nucleoni che vi risiedono. Questa seconda possibilità, basata sui \emph{numeri di occupazione} dello stato, risulta particolarmente utile nel caso dei fermioni dove il principio di esclusione limita i numeri di occupazione ai soli valori ‘0’ ed ‘1’. Nel prosieguo interpreteremo la funzione d’onda in questo modo.

Fatta questa precisazione possiamo proseguire ricordando che, in accordo con i principi generali della meccanica quantistica, la funzione d’onda (\ref{eq:wave-function-nucleon-fermion-gas-normalized}) descrive nel modo più completo possibile lo stato del nucleo per cui è attraverso di essa che si possono calcolare tutte le grandezze fisiche d’interesse. Nel prosieguo deriveremo la \emph{distribuzione energetica cumulativa} dei nucleoni, necessarie per calcolare le \emph{energie medie dei nucleoni} a loro volta necessarie per il calcolo delle \emph{energie di legame}. Vediamo di cosa si tratta.

Per cominciare domandiamoci quale sia \emph{il numero di stati quantomeccanici che possiedono una energia minore o uguale ad un certo prefissato valore}.
\begin{marginfigure}
	\includegraphics{figs/cube-fermion-gas2}
	%%\caption{Fermi surface.}%%
%  \label{fig:goccia1}
\end{marginfigure}
Dato il legame esistente tra energia e quantità di moto, si deve calcolare il numero di stati aventi modulo della quantità di moto minore od uguale ad un certo prefissato valore.

Utilizzando una tecnica impiegata nel caso della radiazione elettromagnetica di cavità, introduce lo \emph{spazio tridimensionale delle quantità di moto} di assi $p_{x}, p_{y}$ e $p_{z}$. Tenendo conto delle relazioni di quantizzazione (\ref{eq:quantized-momentum-fermions-gas}), si comprende immediatamente che i valori delle quantità di moto accessibili al nucleone si collocano nell’ottante positivo dello spazio (data la positività di $n_{x}, n_{y}$ e $n_{z}$ ), sulle intersezioni di un grigliato tridimensionale di passo $\frac{\pi \hslash}{L}$.
A ciascuna di tali intersezioni corrisponde una determinata terna ordinata di numeri quantici $n_{x}, n_{y}$ e $n_{z}$ e, con essa, un determinato stato quantomeccanico (\ref{eq:wave-function-nucleon-fermion-gas-normalized}).


Ciò premesso, si capisce che \textit{il numero di stati quantomeccanici aventi modulo della quantità di moto uguale o inferiore ad un prefissato valore}
$\tilde{p}$ è approssimativamente uguale al quoziente tra il volume dell’ottante di sfera di raggio $\tilde{p}$ ed il volumetto cubico del grigliato
\[
n_{s} = \frac{1}{8} \frac{ \frac{4}{3} \pi \tilde{p}^{3}}{\left( \frac{\pi \hslash}{L} \right)^{3}} = \frac{L^{3}}{6 \pi^{2}\hslash^{3}}\tilde{p}^{3}
\]
dove abbiamo posto $L^{3} = V$.
Un attimo di attenzione chiarisce che tale formula in realtà sovrastima il numero di stati quantomeccanici.
Infatti, tutti gli statti quantomeccanici interni all’ottante sferico di cui sopra, ma giacenti sui tre piani coordinati $p_x=0, p_y=0$ e $p_z=0$, annullano la funzione d’onda (\ref{eq:wave-function-nucleon-fermion-gas-normalized}) e corrispondono pertanto ad un medesimo stato quantico che deve essere contato una volta sola. E’ facile capire che il numero di tali stati è uguale al quoziente tra le aree dei tre quarti di circonferenza di raggio $\tilde{p}$ dell’ottante sferico e l’areola quadrata associata a ciascuno stato quantomeccanico
\[
n_{0} = \frac{3 \frac{1}{4} \pi \tilde{p}^{2}}{\left( \frac{\pi \hslash}{L} \right)^{2}} = \frac{3}{4} \frac{L^{2}}{\pi \hslash^{2}} \tilde{p}^{2} =
\frac{S}{8\pi \hslash^{2}}\tilde{p}^{2}
\]
dove, nell’ultimo passaggio, abbiamo introdotto la superficie della scatola cubica $S = 6 L^{2}$.
Troviamo allora che \textit{ il numero di stati quantici tali che} $0< | p|<\tilde{p}$  in realtà vale
\[
n_{s} = \frac{V}{6 \pi^{2}\hslash^{3}} \tilde{p}^{3} - \frac{S}{8\pi \hslash^{2}}\tilde{p}^{2}
\]
Dato che, secondo il principio di esclusione di Pauli, in ciascuno stato quantistico della funzione d’onda orbitale possono alloggiare al massimo due fermioni identici (uno per ciascuno dei due stati di spin del nucleone descritti da uno spinore che abbiamo omesso) concludiamo che \textbf{il numero di neutroni o protoni identici che possono essere contenuti nel volume nucleare V di superficie S con valore massimo dell’impulso} $P_F$ è dato dalla espressione seguente
\begin{marginfigure}
	\includegraphics{figs/Fermi-surface}
	\caption{Fermi surface.}
%	\label{fig:goccia1}
\end{marginfigure}
\begin{equation}
	n_{F} = \frac{V P_{F}^{3}}{3 \pi^{2}\hslash^{3}} \left( 1 - \frac{3\pi \hslash}{4} \frac{S}{VP_{F}} \right)
	\label{eq:fermions-number-specified-volume-momentum}
\end{equation}

Essa costituisce la formula fondamentale del modello a gas di fermioni, base di ogni successivo sviluppo del modello stesso.
\bigskip

La (\ref{eq:fermions-number-specified-volume-momentum}) permette di stimare immediatamente alcune grandezze fisiche fondamentali del modello.
Trascurando il termine di superficie che si può mostrare essere assi più piccolo di quello di volume, ricaviamo
l’espressione \emph{dell’impulso di fermi in funzione del numero di fermioni nucleari}\sidenote{
	Tenendo conto del fatto che \[\frac{\hslash S}{V} \ll 1\]
}
\begin{equation}
	P_{F} \simeq \hslash \left(\frac{3\pi^{2}n_{F}}{V} \right)^{1/3}
	\label{eq:fermi-momentum-estimate}
\end{equation}
Le ipotesi di \emph{stabilità} del nuclide e dell'occupazione dello \emph{stato di minima energia} si traducono in
\[
	V \simeq \frac{4}{3}\pi r_{0}^{3}A \quad e \quad n_{p}\simeq n_{s} \simeq \frac{A}{2}
\]
per cui usando la (\ref{eq:fermi-momentum-estimate}) possiamo ottenere una stima del valore \textbf{dell’impulso di Fermi} dei protoni e neutroni nucleari
\[
P_{F} = \left( \frac{9\pi}{8} \right)^{1/3} \frac{\hslash}{r_{0}} \simeq 1.52 \frac{\hslash}{1.2 \times \frac{1}{200}\frac{\hslash c}{MeV}} \simeq 254 \, \frac{MeV}{c}
\]
Dunque, \emph{nel nucleo prossimo allo stato fondamentale, esiste una frazione di nucleoni con un impulso piuttosto rilevante}. Inoltre, \emph{l’impulso di Fermi dei protoni e neutroni risulta essere indipendentemente dal tipo e dimensioni del nucleo}, un fatto che non ci sorprende solo a posteriori.
\bigskip

E’ immediato allora stimare la \textbf{energia cinetica di Fermi} dei protoni e neutroni nucleari
\begin{equation}
	E_{F} = \frac{P_{F}^{2}}{2M_{n,p}} \simeq 34 \, MeV
	\label{eq:fermi-kinetic-energy-estimate}
\end{equation}
%Nel caso del nuclepiù semplice ($H$) si avrà che
ed anche la \textbf{profondità della buca di potenziale nucleare} che sarà approssimativamente uguale alla somma della energia di Fermi con la energia media di separazione dei nucleoni che vale circa $8 \, MeV$
(vedi Fig.~\ref{fig:en-legame-graph})
\begin{equation}
	V_{0} \simeq E_{F} + \frac{B}{A} \simeq (34 + 8)\, MeV \simeq 42 \, MeV
   \label{eq:potential-well-fermion-gas-estimate}
\end{equation}
entrambi indipendenti da A \emph{e dunque approssimativamente valide per tutti i nuclei}.

\subsection{L'espressione dell'energia di legame}\label{subsec:energia-di-legame-fermions-gas}

Si noti che l’energia cinetica dei nucleoni non è molto inferiore alla profondità della buca (è inferiore di 8 MeV appunto) per cui \emph{il nucleo è un insieme di nucleoni debolmente legati}.

Nel caso dei nuclei stabili pesanti dobbiamo tenere conto che il numero di neutroni eccede quello dei protoni per cui dalla (\ref{eq:fermi-momentum-estimate}) e (\ref{eq:fermi-kinetic-energy-estimate}) otteniamo \emph{che l’impulso e la energia di Fermi dei neutroni supera quello dei protoni}
\[
P_{F}^n > P_{F}^p \qquad E_{F}^n > E_{F}^p
\]
e così pure per la (\ref{eq:potential-well-fermion-gas-estimate}) \emph{la profondità della buca di potenziale dei neutroni supera quella dei protoni}
\[
V_{0}^n > V_{0}^p
\]
Se ora teniamo conto che i protoni - a causa della carica elettrica che possiedono - sono soggetti ad un potenziale repulsivo coulombiano sostanzialmente apprezzabile solo
quando quello nucleare si azzera, abbiamo che i potenziali complessivi di
neutroni e protoni devono avere l’andamento approssimato mostrato in Figura~\ref{fig:fermi-gas-model-potential-scheme}.

\begin{figure}
	\centering
	\includegraphics{../figs/fermi-gas-model-potential-scheme}
	\caption{Scheme of Fermi gas model potential.}
	\label{fig:fermi-gas-model-potential-scheme}
\end{figure}
\bigskip

Queste considerazioni chiariscono le linee di ragionamento che possono condurre alla costruzione della \emph{formula della energia di legame basata sul modello a gas di fermioni}.

Un \textbf{neutrone porta un contributo medio alla energia di legame nucleare} pari alla differenza tra la profondità della buca di potenziale e la sua energia cinetica media
\[
b_{n} = V_{0} - \langle T_{n} \rangle
\]
Per un protone si deve ragionare allo stesso modo aggiungendovi però la repulsione coulombiana che rende la buca del potenziale totale meno profonda.
La repulsione coulombiana media per protone ha la seguente espressione (8-1 Feynman\cite{FeynLect2})
\[
	\langle V_{coul} \rangle = \frac{1}{Z} \mathlarger{\sum}_{\substack{j,k=1 \\ j \neq k, \,j > k  }}^{Z} \frac{e^{2}}{4 \pi \epsilon_{0}r_{jk}}\]
e dipende dalla disposizione spaziale dei protoni.
Ipotizzando una distribuzione spazialmente uniforme (problema della sfera uniformemente carica) otteniamo (8-1 Feynman)
\[
U_{coul}^{tot} = \frac{3}{5} \frac{e^{2}}{4 \pi \epsilon_{0}} \frac{Z^{2}}{R} = \frac{3}{5} \frac{e^{2}}{4 \pi \epsilon_{0} \hslash c} \hslash c \frac{Z^{2}}{R} = \frac{3}{5} \alpha \hslash c \frac{Z^{2}}{R}
\]
dove $\alpha \simeq \frac{1}{137}$ è la \emph{costante adimensionale di struttura fina} e $R$ è il raggio della distribuzione sferica di carica ovvero il raggio nucleare.
Invocando un ragionamento già fatto per il modello a goccia quando si è parlato del termine coulombiano (si veda (\ref{eq:work-to-seperate-uniformly-distributed-protons})) nel nostro modello la precedente si riscrive come
\[
U_{coul}^{tot} = \frac{3}{5} \alpha \hslash c \frac{Z (Z-1)}{R} \ \cdot
\]
Possiamo ora scrivere il \emph{contributo medio del protone alla energia di legame nucleare} $b_{p}$\sidenote
{
Osserviamo che $ b_n$ e $ b_p$ sono stati stimati teoricamente a partire dal modello a gas di Fermi e non tramite
considerazioni sul difetto di massa come in precedenza. Di fatto si vuole migliorare il precedente modello fenomenologico
(a goccia) cercando di ricavare la dipendenza di $ B$ da $ A$ e $ Z$.
}
\begin{gather*}
    \langle U^{tot}_{coul} \rangle = \frac{U^{tot}_{coul}}{Z}\\
    b_{p} = V_{0} - \langle T_{p} \rangle  - \frac{1}{Z} U^{tot}_{coul}
\end{gather*}
Tenendo ora conto espressione e dell’analoga per i neutroni possiamo comporre la seguente \textbf{espressione della energia di legame nucleare}
\begin{equation}
	B = N b_{n} + Z b_{p} = A V_{0} - N \langle T_{n} \rangle - Z \langle T_{p} \rangle - \frac{3}{5} \alpha \hslash c \frac{Z(Z-1)}{R}
	\label{eq:binding-energy-fermi-gas-model}
\end{equation}


\subsection{Il calcolo dell'energia di legame}\label{sec:calcolo-energia-legame-fermions-gas}

Per sviluppare l'enegia di legame è ora necessario calcolare le energie cinetiche medie dei neutroni e protoni nucleari.
Si ha
\[
\langle T \rangle = \frac{\sum_{i}T_{i}n_{i}'}{\sum_{i}n_{i}'} \quad , \quad n_{i}' \circeq \Delta_{i}n = n_{i+1}- n_{i} \quad
i = 0, \dots, \bar{n}
\]
dove $\bar{n}$ è il numero di protoni/neutroni propri del livello di Fermi.
Ora, immaginando un addensamento di strati energetici tra $0$ ed il livello di fermi $E_{F}$, possiamo effettuare il passaggio al continuo
\begin{equation}
	\langle T \rangle = \fraclarg{\int_{0}^{P_{F}} \frac{p^{2}}{2m} \, dn}{\int_{0}^{P_{F}}  \, dn }
	\label{eq:average-kinetic-energy-nucleons}
\end{equation}
dove $dn$ indica il numero di neutroni/protoni con impulso compreso tra $p$ e $p+dp$.

Il differenziale della (\ref{eq:fermions-number-specified-volume-momentum}) fornisce tale numero di neutroni/protoni
\[
dn = \frac{V}{\pi^{2}\hslash^{3}} p^{2}dp - \frac{S}{2 \pi \hslash^{2}} pdp
\]
che può essere sostituito nella (\ref{eq:average-kinetic-energy-nucleons}) ottenendo
\begin{align*}
    \langle T \rangle &= \fraclarg{\int_{0}^{P_{F}} \frac{p^{2}}{ 2m}\left(\frac{V}{\pi^{2}\hslash^{3}} p^{2}dp - \frac{S}{2 \pi \hslash^{2}} pdp \right)  }{\frac{V}{3\pi^{2}\hslash^{3}} P_{F}^{3} - \frac{S}{4 \pi \hslash^{2}} P_{F}^{2} }
    = \fraclarg{\frac{VP_{F}^{5}}{10m\pi^{2}\hslash^{3}} - \frac{SP_{F}^{4}}{16m\pi \hslash^{2}}}{\frac{V}{3\pi^{2}\hslash^{3}} P_{F}^{3} - \frac{S}{4 \pi \hslash^{2}} P_{F}^{2}}\\
    &= \fraclarg{\frac{VP_{F}^{5}}{10m \pi^{2}\hslash^{3}}\left( 1 - \frac{10m\pi^{2}\hslash^{3}}{VP_{F}^{5}} \frac{SP_F^{4}}{16 m \pi \hslash^{2}} \right)}{\frac{VP_{F}^{3}}{3\pi^{2}\hslash^{3}}\left( 1 - \frac{3\pi^{2}\hslash^{3}}{VP_{F}^{3}} \frac{S}{4 \pi \hslash^{2}}P_{F}^{2} \right)}
	= \frac{3}{10 \,m}P_{F}^{2} \fraclarg{1 - \frac{5\pi \hslash}{8} \frac{S}{VP_{F}}}{1 - \frac{3\pi \hslash}{4} \frac{S}{VP_{F}}}
\end{align*}
Tramite un'espansione in serie di Taylor al primo ordine
\begin{gather*}
    a = \frac{5\pi}{8 P_{F}} \quad b = \frac{3\pi }{4 P_{F}} \quad x = \hslash\frac{S}{V} \ (x \ll 1)\\
    \frac{1-ax}{1-bx} = 1 + (b-a)x + O(x^{2})
\end{gather*}
otteniamo
\begin{align}
    \langle T \rangle &\simeq  \frac{3P_{F}^{2}}{10 \, m} \left( 1 - \frac{\frac{5\pi \hslash}{8}S}{VP_{F}}+ \frac{3\pi \hslash}{4} \frac{S}{V P_{F}} \right) \nonumber\\
     &= \frac{3}{10 \, m}P_{F}^{2} \left( 1 + \frac{\pi\hslash S}{8 V P_{F}} \right)
	\label{eq:approximated-kinetic-energy-nucleons}
\end{align}
Tenendo conto solo del termine dominante, possiamo migliorare la stima (\ref{eq:fermi-kinetic-energy-estimate}) del \emph{valore della energia cinetica media dei neutroni e protoni nucleari}
\[
\langle T \rangle \simeq \frac{3}{10 \, m} P_{F}^{2} \simeq \frac{3}{5}E_{F} \simeq \frac{3 \times 34}{5} \simeq 21 \, MeV
\]
Ritornando alla (\ref{eq:approximated-kinetic-energy-nucleons}) è chiaro che si vuole disporre di una espressione della energia cinetica media dove compaiano i parametri nucleari $(N, Z, V, S)$ e non il valore dell’impulso di Fermi.
Ciò significa che conviene eliminare la variabile $P_{F}$ nella (\ref{eq:approximated-kinetic-energy-nucleons}) utilizzando la (\ref{eq:fermions-number-specified-volume-momentum}), che nel contesto delle approssimazioni utilizzate\sidenote{
Per arrivare a (\ref{eq:approximated-kinetic-energy-nucleons}) si è sviluppato in serie di Taylor. Tale serie risulta convergente nel caso in cui $| b| < \frac{1}{| x|}$, cioè se
	\[
		\frac{3\pi \hslash}{4} \frac{S}{V P_{F}} \ll 1
	\]
	condizione che permette di compiere l'approssimazione svolta.
} si approssima a
\[
n_{F} =  \frac{VP_{F}^{3}}{3 \pi^{2}\hslash^{3}} \left( 1 -  \frac{3\pi \hslash}{4} \frac{S}{V P_{F}} \right) \simeq \frac{VP_{F}^{3}}{3 \pi^{2} \hslash^{3}}
\]
da cui
\begin{equation}
	P_{F} = \left( \frac{3\pi^{2}\hslash^{3}n_{F}}{V} \right)^{1/3}
	\label{eq:fermi-momentum-in-terms-of-fermi-number}
\end{equation}
Sostituendo (\ref{eq:fermi-momentum-in-terms-of-fermi-number}) in (\ref{eq:approximated-kinetic-energy-nucleons})
\[
\langle T \rangle \simeq \frac{3}{10 \, m} \left( \frac{3\pi^{2}\hslash^{3}n_{F}}{V} \right)^{2/3} + \frac{3}{80 \, m} \frac{\pi \hslash S}{V}\left( \frac{3\pi^{2}\hslash^{3}n_{F}}{V} \right)^{1/3}
\]
da cui, infine, \emph{l’espressione cercata della energia cinetica media dei fermioni in funzione dei parametri nucleari}
\[
\langle T \rangle \simeq \frac{9\hslash^{2}}{10 \, m} \left( \frac{\pi^{4}n_{F}^{2}}{3 V^{2}}\right)^{1/3}  + \frac{9\hslash^{2}}{80 \, m}\left( \frac{\pi^{5}n_{F}}{9V^{4}} \right)^{1/3}S
\]
Sostituendo questa espressione nella (\ref{eq:binding-energy-fermi-gas-model}) ci avviamo verso l’espressione finale della energia di legame
\begin{align*}
	B \simeq A V_{0} &- N \left[ \frac{9\hslash^{2}}{10\, m} \left( \frac{\pi^{4}N^{2}}{3 V^{2}} \right)^{1/3} + \frac{9\hslash^{2}}{80 \, m}\left(  \frac{\pi^{5}N}{9V^{4}} \right)^{1/3}S \right] \\
	& -Z \left[ \frac{9\hslash^{2}}{10\, m} \left( \frac{\pi^{4}Z^{2}}{3 V^{2}} \right)^{1/3} + \frac{9\hslash^{2}}{80 \, m}\left(  \frac{\pi^{5}Z}{9V^{4}} \right)^{1/3}S \right] \\
	& - \frac{3}{5} \alpha \hslash c \frac{Z(Z-1)}{R} \\
	= AV_{0} & - \frac{9\hslash^{2}}{80 \, m}\left( \frac{\pi^{5}}{9} \right)^{1/3} \frac{S(N^{4/3}+Z^{4/3})}{V^{4/3}} \\
	&- \frac{9\hslash^{2}}{10 \, m} \left( \frac{\pi^{4}}{3} \right)^{1/3} \frac{N^{5/3}+Z^{5/3}}{V^{2/3}} - \frac{3}{5} \alpha \hslash c \frac{Z(Z-1)}{R}
\end{align*}
dove abbiamo introdotto una unica massa per protone e neutrone.
Ora dobbiamo ricordare che nel nucleo $V, R, N$ e $Z$ sono variabili correlate
\[
V = \frac{4}{3} \pi r_{0}^{3}A \qquad S = 4\pi r_{0}^{2}A^{2/3} \qquad R = r_{0}A^{1/3}
\]
per cui sostituendo nella precedente
\begin{align*}
	= AV_{0} &- \frac{9 \hslash^{2}}{80 \, m} \left( \frac{\pi^{5}}{9} \right)^{1/3} \frac{4 \pi r_{0}^{2}A^{2/3}(N^{4/3}+Z^{4/3})}{\left( \frac{4}{3} \pi r_{0}^{3} A\right)^{4/3}} \\
	&- \frac{9 \hslash^{2}}{10 \, m} \left( \frac{\pi^{4}}{3} \right)^{1/3} \frac{N^{5/3}+Z^{5/3}}{\left( \frac{4}{3} \pi r_{0}^{3} A\right)^{2/3}} - \frac{3}{5} \alpha \hslash c \frac{Z(Z-1)}{r_{0}A^{1/3}} \\
	= AV_{0} &- \frac{9 \hslash^{2}}{80 \, m} \left( \frac{\pi^{5}}{9} \right)^{1/3} \frac{3^{4/3}}{ 4^{1/3}\pi^{1/3}r_{0}^{2}} \frac{N^{4/3}+Z^{4/3}}{A^{2/3}} \\
	&- \frac{9 \hslash^{2}}{10 \, m} \left( \frac{\pi^{4}}{3} \right)^{1/3} \frac{3^{2/3}}{ 4^{2/3}\pi^{2/3}r_{0}^{2}}\frac{N^{5/3}+Z^{5/3}}{ A^{2/3}} - \frac{3}{5} \alpha \hslash c \frac{Z(Z-1)}{r_{0}A^{1/3}} \\
	= AV_{0} &- \frac{9 \hslash^{2}}{80 \, m} \left( \frac{9\pi^{4}}{4} \right)^{1/3} \frac{3^{4/3}}{ 4^{1/3}\pi^{1/3}r_{0}^{2}} \frac{N^{4/3}+Z^{4/3}}{A^{2/3}} \\
	&- \frac{9 \hslash^{2}}{10 \, mr_{0}^{2}} \left( \frac{3\pi^{2}}{16} \right)^{1/3} \frac{N^{5/3}+Z^{5/3}}{ A^{2/3}} - \frac{3}{5} \alpha \hslash c \frac{Z(Z-1)}{r_{0}A^{1/3}}
\end{align*}
da cui infine
\begin{align}
	B = A V_{0} &- \frac{9\hslash^{2}}{80 mr_{0}^{2}} \left( \frac{3\pi^{2}}{2} \right)^{2/3}\frac{N^{4/3}+Z^{4/3}}{A^{2/3}} \nonumber\\
	&- \frac{3\hslash^{2}}{10 mr_{0^{2}}}\left( \frac{9\pi}{4} \right)^{2/3} \frac{N^{5/3}+Z^{5/3}}{A^{2/3}} - \frac{3}{5} \alpha \hslash c \frac{Z(Z-1)}{r_{0}A^{1/3}}
	\label{eq:intermediate-calc-binding-energy-fermi-gas}
\end{align}
Per avere questa espressione in una forma confrontabile con l’espressione empirica della energia di legame nucleare conviene passare alle variabili
\[
A=(N+Z) \quad\delta = (N-Z) \qquad N = \frac{A+\delta}{2} \quad Z = \frac{A- \delta}{2}
\]
tenendo anche presente che la variabile $\delta$, nel caso dei nuclei stabili, tende ad essere piccola.

Possiamo allora approssimare con il seguente sviluppo in serie, trattenendo anche i termini del secondo ordine
\[
(1 + x)^{\alpha} \simeq 1 + \alpha x + \frac{1}{2} \alpha (\alpha - 1)x^{2}
\]
I termini in N, Z e A della (\ref{eq:intermediate-calc-binding-energy-fermi-gas}) prendono allora le seguenti forme
\begin{align*}
	\frac{N^{5/3}+Z^{5/3}}{ A^{2/3}} & = \frac{(A+\delta)^{5/3} + (A-\delta)^{5/3}}{2^{5/3} A^{2/3}} = \frac{A}{2^{5/3}} \left[ \left( 1+\frac{\delta}{A} \right)^{5/3} +\left( 1 - \frac{\delta}{A} \right)^{5/3} \right] \\
	& \simeq \frac{A}{2^{5/3}} \left[ 1 + \frac{5}{3}\left( \frac{\delta}{A} \right) + \frac{5}{9} \left( \frac{\delta}{A} \right)^{2} + 1 - \frac{5}{3} \left( \frac{\delta}{A} \right) + \frac{5}{9} \left( \frac{\delta}{A} \right)^{2} \right] \\
	& = \frac{A}{2^{2/3}}  \left[ 1+\frac{5}{9} \left( \frac{\delta}{A} \right)^{2} \right]
	\simeq \frac{A}{2^{2/3}} \left[ 1 + \frac{5}{9} \left( \frac{N-Z}{A} \right)^{2} \right]
\end{align*}
\begin{align*}
	\frac{N^{4/3}+Z^{4/3}}{ A^{2/3}} & = \frac{(A+\delta)^{4/3} + (A-\delta)^{4/3}}{2^{4/3} A^{2/3}} = \frac{A}{2^{4/3}} \left[ \left( 1+\frac{\delta}{A} \right)^{4/3} +\left( 1 - \frac{\delta}{A} \right)^{4/3} \right] \\
	& \simeq \frac{A}{2^{4/3}} \left[ 1 + \frac{4}{3}\left( \frac{\delta}{A} \right) + \frac{2}{9} \left( \frac{\delta}{A} \right)^{2} + 1 - \frac{4}{3} \left( \frac{\delta}{A} \right) + \frac{2}{9} \left( \frac{\delta}{A} \right)^{2} \right] \\
	& = \frac{A}{2^{1/3}}  1+\frac{5}{9} \left( \frac{\delta}{A} \right)^{2}
	\simeq \frac{A}{2^{2/3}} \left[ 1 + \frac{2}{9} \left( \frac{N-Z}{A} \right)^{2} \right]
\end{align*}
Sostituendo $N=A-Z$ otteniamo infine le seguenti espressioni
\begin{align*}
	\frac{N^{5/3}+Z^{5/3}}{ A^{2/3}} &\simeq \frac{A}{2^{2/3}} \left[ 1 + \frac{5}{9} \left( \frac{A-2Z}{A} \right)^{2} \right]\\
	\frac{N^{4/3}+Z^{4/3}}{ A^{2/3}} &\simeq \frac{A}{2^{2/3}} \left[ 1 + \frac{2}{9} \left( \frac{A-2Z}{A} \right)^{2} \right]
\end{align*}
che sostituite nella (\ref{eq:intermediate-calc-binding-energy-fermi-gas}) permettono di sviluppare ulteriormente
l’espressione della energia di legame
\marginnote{Il fattore numerico evidenziato è diverso nei calcoli del professore.}
\begin{align*}
	B \simeq A V_{0} &- \frac{9\hslash^{2}}{80 mr_{0}^{2}}\left( \frac{3\pi^{2}}{2} \right)^{2/3} \frac{A^{2/3}}{2^{1/3}}  \left[ 1+ \frac{2}{9} \left( \frac{A-2Z}{A} \right)^{2} \right] \\
	&- 3 \frac{\hslash^{2}}{10 m r_{0}^{2}} \left( \frac{9\pi}{4} \right)^{2/3} \frac{A}{2^{2/3}}  \left[ 1 + \frac{5}{9} \left( \frac{A-2Z}{A} \right)^{2} \right] - \frac{3 \alpha \hslash c}{5 r_{0}} \frac{Z(Z-1)}{A^{1/3}} \\
	AV_{0} &- \frac{9\hslash^{2}}{80mr_{0}^{2}} \left( \frac{3\pi^{2}}{2\sqrt{ 2 }} \right)^{2/3} A^{2/3} - \frac{\hslash^{2}}{\textcolor{cyan}{40} m r_{0}^{2}} \left( \frac{3\pi^{2}}{2\sqrt{ 2 }} \right)^{2/3} \frac{(A-2Z)^{2}}{A^{4/3}} \\
	&- \frac{3\hslash^{2}}{10mr_{0}^{2}} \left( \frac{9\pi}{8} \right)^{2/3}A - \frac{\hslash^{2}}{6mr_{0}^{2}} \left( \frac{9\pi}{8} \right)^{2/3} \frac{(A-2Z)^{2}}{A} - \frac{3 \alpha \hslash c}{5 r_{0}} \frac{Z(Z-1)}{A^{1/3}}
\end{align*}
che ora riscriviamo nella sua forma finale
\begin{equation}
	\boxed{
	\begin{aligned}
		B &\simeq \left[ V_{0} - \frac{3\hslash^{2}}{10mr_{0}^{2}} \left( \frac{9\pi}{8} \right)^{2/3} \right] A - \left[ \frac{9\hslash^{2}}{80mr_{0}^{2}} \left( \frac{3\pi^{2}}{2\sqrt{ 2 }} \right)^{2/3} \right]A^{2/3}
		- \frac{3 \alpha \hslash c}{5 r_{0}} \frac{Z(Z-1)}{A^{1/3}}  \\
		&- \left[ \frac{\hslash^{2}}{6mr_{0}^{2}} \left( \frac{9\pi}{8} \right)^{2/3}  \right]\frac{(A-2Z)^{2}}{A}
		- \left[ \frac{\hslash^{2}}{\textcolor{cyan}{40} m r_{0}^{2}} \left( \frac{3\pi^{2}}{2\sqrt{ 2 }} \right)^{2/3} \right] \frac{(A-2Z)^{2}}{A^{4/3}}
	\end{aligned}
	}
	\label{eq:binding-energy-expression-fermi-gas-model}
\end{equation}
%\begin{align}
%		B &\simeq \left[ V_{0} - \frac{3\hslash^{2}}{10mr_{0}^{2}} \left( \frac{9\pi}{8} \right)^{2/3} \right] A - \left[ \frac{9\hslash^{2}}{80mr_{0}^{2}} \left( \frac{3\pi^{2}}{2\sqrt{ 2 }} \right)^{2/3} \right]A^{2/3}
%	- \frac{3 \alpha \hslash c}{5 r_{0}} \frac{Z(Z-1)}{A^{1/3}} \nonumber \\
%	&- \left[ \frac{\hslash^{2}}{6mr_{0}^{2}} \left( \frac{9\pi}{8} \right)^{2/3}  \right]\frac{(A-2Z)^{2}}{A}
%	- \left[ \frac{\hslash^{2}}{\textcolor{cyan}{40} m r_{0}^{2}} \left( \frac{3\pi^{2}}{2\sqrt{ 2 }} \right)^{2/3} \right] \frac{(A-2Z)^{2}}{A^{4/3}}
%	\label{eq:binding-energy-expression-fermi-gas-model}
%\end{align}


Abbiamo così ottenuto l’espressione della \textbf{energia di legame nucleare nell’ambito del modello a gas di fermioni}.

Si noti che i coefficienti dei termini in $A$ e $Z $ dipendono - oltre che da costanti fisiche ben note - da due soli parametri nucleari:
la profondità della buca di potenziale $V_{0}$ ed il raggio del nucleone $r_{0}$ confermando la notevole capacità predittiva del modello.
Fatto ancor più rilevante, vengono correttamente previsti i termini di volume, superficie, coulombiano e asimmetria della 
espressione del modello a goccia (\ref{eq:weizsacker-formula}):
\begin{align*}
	a_{v} &= V_{0} - 3 \frac{\hslash^{2}}{10 m r_{0}^{2}} \left( \frac{9\pi}{8} \right)^{2/3} \\
	a_{s} &= \frac{9\hslash^{2}}{80 m r_{0}^{2}} \left(  \frac{3\pi^{2}}{2\sqrt{ 2 }} \right)^{2/3} \\
	a_{c} &= \frac{3 \alpha \hslash c}{5 r_{0}} \\
	a_{a} &= \frac{\hslash^{2}}{6 m r_{0}^{2}} \left(\frac{9\pi}{8} \right)^{2/3}
\end{align*}
Tali termini vengono così ricondotti ai principi generali della meccanica quantistica e possono ora fondarsi su di un meccanismo fisico di validità generale.

Ad esempio, tenendo presente il procedimento seguito, possiamo verificare che i termini di volume e superficie sono essenzialmente dovuti alla quantizzazione degli stati dei nucleoni a seguito del loro confinamento all’interno del volume nucleare.

Il termine coulombiano è trattato allo stesso modo del modello a goccia.

Il termine di asimmetria, invece, è ricondotto al principio di esclusione di Pauli e dunque ad uno dei fatti più profondi della meccanica quantistica: nello stato nucleare di minima energia risulta energeticamente conveniente
legare tra loro numeri confrontabili di protoni e neutroni piuttosto che soli neutroni, spiegando così una delle proprietà più controintuitive dei nuclei.
Se si tiene poi conto della mutua repulsione dei protoni si capisce anche la tendenza a legare numeri di neutroni eccedenti rispetto ai protoni.

Meno fortuna ha il termine di accoppiamento.
D’altra parte gli ingredienti immessi nel modello non potevano in nessun modo condurre al risultato di una maggiore stabilità dei nuclei pari-pari, un fatto che richiede evidentemente un potenziale più appropriato che tenga conto anche dello spin dei nucleoni.

In generale, il successo del modello a gas di fermioni dimostra che
\emph{la maggior parte delle proprietà dei nuclei in prossimità dello stato fondamentale sono legate alle proprietà
collettive dei nucleoni in condizioni di forte degenerazione piuttosto che alle specifiche proprietà della forza nucleare}
la quale viene riassunta in una semplice forza di contenimento dei nucleoni.

Si può mettere alla prova in modo ancora più severo il modello calcolando esplicitamente i valori dei diversi termini e confrontandoli con quelli empirici.
Sostituendo i valori numerici\sidenote{
Usando i seguenti valori:
\begin{gather*}
    r_0 = 1.25 \, fm \\
	V_0 = 42 \,MeV \\
	m = 940 \, MeV/c^2
\end{gather*}
} otteniamo ( i valori sono in $MeV$)
\[
	a_v \simeq 22 \quad
	a_s \simeq 14.6 \quad
	a_c \simeq 0.070 \quad
	a_a \simeq 10.5
\]
Ad esclusione del termine di accoppiamento di cui abbiamo detto, l’accordo è relativamente soddisfacente
(vedi set di valori (\ref{eq:typical-values-drop-model})) ed è assai degno di nota il fatto che tale accordo sia
ottenuto introducendo due soli parametri liberi (la profondità della buca di potenziale nucleare $V_0$
ed il raggio del nucleone $r_0$).
%%%%%%%%%%%%%%%%%%%%%%%%%%%%%%%%%%%%%%%%%%%%%%%%%%%%%%%%%%%%%%%
\section{Il modello a Shell}\label{sec:il-modello-a-shell}

Dobbiamo ora domandarci quale possa essere il motivo del successo solo parziale del modello a gas di Fermi nella descrizione della energia di legame dei nuclei stabili.

Il suo successo è certamente dovuto al fatto che riconosce il \textbf{ruolo determinante giocato dalla meccanica quantistica} che si manifesta
\begin{itemize}
	\item nella discretizzazione degli stati quantomeccanici e delle loro energie;
	\item nella possibilità di collocare in ciascuno stato quantomeccanico un solo nucleone (principio di esclusione di Pauli).
\end{itemize}

Il suo insuccesso, invece, è probabilmente dovuto alla eccessiva \textbf{semplificazione delle forze nucleari} in gioco poiché si ipotizza che le complesse interazioni che ogni nucleone ha con i suoi vicini possano riassumersi in una \textbf{interazione efficace} \emph{dipendente dalla posizione, nulla all’interno del volume nucleare e di modulo assai elevato e centripeta sulla sua superficie} (buca sferica di potenziale). Tali assunzioni devono essere riviste criticamente.

\subsection{Il potenziale nucleare medio}\label{sec:potenziale-nucleare-medio}

Come già osservato quando costruimmo il modello nucleare a gas di fermioni, la risultante delle forze percepita dal singolo nucleone si dovrebbe riassumere in una forza mediamente nulla all’interno del volume nucleare e mediamente non nulla e centripeta in prossimità della superficie nucleare.
Dunque, un potenziale medio a forma di scatola – se assumiamo la descrizione più drastica – o di forma fisicamente più plausibile con variazioni di potenziale meglio raccordate.

Rimandando per ora il problema della forma concreta del potenziale, sappiamo che - sulla base della meccanica quantistica – gli stati quantomeccanici e le energie dei nucleoni soggetti ad un tale potenziale risulteranno discretizzati.
D’altra parte, il principio di Pauli permette di occupare ciascun stato quantomeccanico con un solo nucleone (qualora si tenga conto anche dello spin) per cui su ogni livello energetico potranno risiedere solo un ben definito numero di neutroni e protoni.
Dato che un nucleo nello stato fondamentale possiede la minima energia, deduciamo che i neutroni ed i protoni andranno a riempire dal basso i diversi livelli energetici nucleari fino a raggiungere due livelli massimi detti \emph{livelli di Fermi} in completa analogia con un gas di fermioni degenere alle bassissime temperature.

Ora, immaginiamo che un neutrone (protone) del nucleo collida con un secondo neutrone (protone).
A seguito della collisione il neutrone (protone) dovrebbe variare la propria energia ma, essendo gli stati quantomeccanici dei livelli energetici completamente occupati, tale variazione avverrà solo nel caso in cui l’energia scambiata sia sufficiente a farlo saltare oltre il livello di Fermi (fenomeno chiamato ‘Pauli blocking’).
Dato che nella maggioranza dei casi ciò non avviene, \emph{il moto del neutrone (protone) risulterà essenzialmente imperturbato} come se gli altri neutroni (protoni) non esistessero affatto e si svilupperà seguendo ‘traiettorie’ regolari ben diverse dalle traiettorie degli atomi/molecole di un gas classico continuamente spezzate dalle reciproche collisioni.

A causa del `Pauli blocking’(vedi sidenote pag~\pageref{siden:pauli-blocking}) dunque, \textbf{l’interazione efficace che governa il moto dei nucleoni nel nucleo non è di tipo stocastico ma piuttosto assimilabile ad una forza posizionale descrivibile quindi da un potenziale}.
\bigskip
\begin{marginfigure}
	\includegraphics{figs/energy-level-shell1}
	\caption{Energetic scheme of a nucleon in the mimimum energy state. The highest-energy state is Fermi level.}
	\label{fig:energy-level-shell1}
\end{marginfigure}
Verificata la esistenza di un potenziale nucleare si apre il problema della sua forma che non è detto possa ridursi alla buca sferica di potenziale assunta dal modello a gas di fermioni.
Benchè capace di riprodurre alcune importanti proprietà del nucleo, questa ipotesi deve essere sottoposta a verifica sperimentale. \emph{In che modo, allora, si può determinare la forma del potenziale nucleare}?

La forma del potenziale determina il numero di stati quantomeccanici di ogni possibile livello energetico (ovvero il grado di degenerazione) ed anche la loro spaziatura.
D’altra parte il principio di esclusione lascia risiedere un solo nucleone in ogni stato quantomeccanico e dunque un numero di nucleoni pari al grado di degenerazione in ogni livello energetico.
Ciò significa che contando i nucleoni sui diversi livelli energetici possiamo ottenere informazioni stringenti sulla forma del potenziale nucleare. \emph{In che modo allora si possono contare i protoni ed i neutroni alloggiati nei livelli energetici nucleari}?

Sappiamo che in un nucleo nello stato fondamentale i neutroni ed i protoni vanno ad occupare tutti gli stati quantomeccanici dei diversi livelli energetici fino a raggiungere i rispettivi livelli di Fermi.
\begin{marginfigure}
	\includegraphics{figs/energy-level-shell2}
	\caption{Energetic scheme of a nucleon in the mimimum energy state. The highest-energy state is Fermi level.}
	\label{fig:energy-level-shell2}
\end{marginfigure}
Dato un nucleo di $N$ neutroni e $Z$ protoni, l’energia che dobbiamo fornire ai neutroni per estrarne uno eguaglia la differenza tra la profondità energetica della buca di potenziale e l’energia del livello di Fermi dei neutroni, ovvero $V_{0}-E_{l}$ (vedi Figura~\ref{fig:energy-level-shell1}). Consideriamo allora l’isotopo con $N+1$ neutroni e $Z$ protoni e domandiamoci ancora una volta quanta energia si debba fornire ai neutroni per estrarne uno dal nucleo. Ci sono due possibili risposte:
\begin{itemize}
	\item se il livello di Fermi dei neutroni del nucleo ($N, Z$) era incompleto, il neutrone in più andrà a collocarsi sullo stesso livello energetico (figura \ref{fig:energy-level-shell1}) e l’energia necessaria sarà ancora $V_{0}-E_{l}$;
    \item se il livello di Fermi dei neutroni del nucleo ($N, Z$) era completo il neutrone in più andrà a collocarsi sul livello energetico successivo e l’energia necessaria avrà un valore inferiore ovvero $V_0 - E_{l+1}$ (vedi Figura \ref{fig:energy-level-shell2}).
\end{itemize}

Giungiamo allora alla conclusione che \emph{le energie di separazione del neutrone in una serie isotopica e del protone in una serie isotonica devono subire bruschi salti in basso in corrispondenza del completamento dei rispettivi livelli energetici nucleari}.
\bigskip

I dati sperimentali delle energie di separazione dei neutroni e protoni nucleari confermano l’andamento previsto.
Nei grafici in Figura~\ref{fig:neutron-separation-energy8-20-82} a lato è infatti possibile osservare diversi salti verso il basso della energia di
separazione del neutrone nel caso delle serie isotopiche dell’Ossigeno, del Calcio e del Piombo che permettono di esplorare
rispettivamente numeri bassi, medi ed elevati di neutroni nucleari.
\begin{marginfigure}
	\includegraphics{figs/neutron-separation-energy8-20-82}
	\caption{Energetic scheme of a nucleon in the mimimum energy state. The highest-energy state is Fermi level.}
	\label{fig:neutron-separation-energy8-20-82}
\end{marginfigure}
Si vedono chiaramente i completamenti dei livelli energetici nucleari dei neutroni in corrispondenza dei numeri $8$
(ossigeno), $20, 28$ (calcio) e $126$ (piombo).
Dati analoghi mostrano altri completamenti in corrispondenza dei numeri $50$ e $82$.

La stessa serie di numeri può essere ottenuta anche per i protoni dalla analisi dei dati sperimentali sulle serie isotoniche.

In sintesi possiamo affermare che
\emph{i dati sperimentali sulla energia di separazione dei neutroni e dei protoni indicano che i  livelli energetici nucleari (‘shell’ nucleari)
	si completano in corrispondenza dei numeri 8, 20, 28, 50, 82 e 126 detti numeri magici}.
\bigskip

Si pone ora il problema di stabilire se il potenziale nucleare a forma di \textbf{buca sferica infinita} sia in grado di riprodurre i numeri magici nucleari. Premesso che già sappiamo che questo potenziale non è adeguato poiché prevede una energia di separazione infinita per neutroni e protoni, da un punto di vista generale si tratta di \textbf{risolvere l’equazione di Schrödinger} trovando le \textbf{autofunzioni} e gli \textbf{autovalori} dell’\textbf{operatore hamiltoniano} costruito con il potenziale
\[
V(r) =
\begin{cases}
	0 \qquad \ se \quad r <R \\
	+ \infty \quad se \quad r > R
\end{cases}
\]
Adottato un \emph{sistema di coordinate sferiche} e tenuto conto che l’hamiltoniano non contiene termini dipendenti dal tempo, l’equazione di Schrödinger può essere risolta sostituendo la seguente espressione generale della funzione d’onda separata nelle variabili spaziali e nella variabile temporale
\[
\psi(\bm{r},t) = \varphi(r, \theta, \phi) e^{ \frac{i}{\hslash} Et}
\]
Dato che il potenziale non dipende dalle variabili angolari, la parte spaziale della funzione d’onda può essere a sua volta separata in due parti: una dipendente dalle variabili angolari ed una dalla variabile radiale. Svolgendo i calcoli si trova che la prima risulta essere una delle \emph{autofunzioni dell’operatore momento angolare}, ovvero una delle armoniche sferiche dipendente dai numeri quantici $l$ ed $m$, mentre la seconda una funzione di Bessel di ordine $l$
\begin{equation}
	\varphi (r, \theta, \phi) = j_{l} (kr) Y_{l,m}(\theta,\phi)
	\label{eq:spatial-part-wave-function}
\end{equation}
Ora, affinché la funzione d’onda possa annullarsi sulla superficie del nucleo ovvero in $r=R$ si deve avere
\[
j_{l}(kR) = 0
\]
che impone la condizione
\[
kR = z_{l,n}
\]
dove, $z_{l,n} \ (n= 1,2,3,\dots)$ \emph{descrive la serie infinita di valori crescenti dell’argomento che annullano la funzione di Bessel} $j_{l}(z)$ di ordine $l$ ed $n$ viene detto \emph{numero quantico principale}. Tale condizione conduce evidentemente alla \emph{quantizzazione del modulo del vettore d’onda} e della \emph{quantità di moto}
\[
k = \frac{1}{R} z_{l,n} \qquad p = \frac{\hslash}{R}z_{l,n}
\]
Dato che nel problema in esame l’energia possiede solo il termine cinetico ($U=0$ all’interno del volume sferico), le precedenti condizioni determinano la \emph{quantizzazione della energia degli stati quantici della buca di potenziale sferica infinita}
\begin{equation}
	E = \frac{\hslash^{2}k^{2}}{2m} = \frac{\hslash^{2}}{2mR^{2}}z_{l,n}^{2}
	\label{eq:energy-levels-infinite-spherical-potential-well}
\end{equation}
Da questa espressione si evince che
\begin{itemize}
	\item l’energia dipende dal numero quantico del momento angolare $l$ e dal numero quantico principale  $n$ e per ogni fissato $l$ (ordine della funzione di bessel) aumenta con $n$ (termine ennesimo della serie di zeri);
	\item ad ogni valore della energia, ovvero ad ogni coppia di valori di $l$ ed $n$, corrispondono $(2l+1)$ funzioni d’onda
	diverse nella sola parte angolare: $Y_{l,l} \  Y_{l, -l+1} \dots Y_{l,l-1} \ Y_{l,l}$ , fatto che si riassume dicendo
	che il livello energetico $l, n$ ha una degenerazione di ordine $(2l+1)$.
\end{itemize}
\begin{marginfigure}
	\includegraphics{figs/energetic-levels-infinite-potential-well}
	\caption{Energetic levels relative to infinite potential well.}
	\label{fig:energetic-levels-infinite-potential-well}
\end{marginfigure}
Introduciamo ora la notazione atomica dove il numero quantico $n$ è indicato esplicitamente al primo posto, mentre il numero quantico $l$ è indicato al secondo posto per mezzo della seguente corrispondenza
\[
l = 0 \to s \quad l = 1 \to p \quad l = 2 \to d \quad l = 3 \to f
\quad l = 4 \to g \quad \dots
\]
Tenuto conto della tabella sottostante che riporta gli zeri delle funzioni di Bessel, dalla formula (\ref{eq:energy-levels-infinite-spherical-potential-well}) otteniamo i livelli energetici della buca di potenziale sferica infinita indicati nella figura~\ref{fig:energetic-levels-infinite-potential-well}: nella prima colonna di destra è indicata la degernerazione in $m$ del livello energetico, nella seconda colonna si calcola il numero di nucleoni contenuti nel livello (due stati di spin per ogni stato orbitale), nella terza colonna è indicato il numero cumulativo di nucleoni. Raggruppando i livelli energetici vicini non risolvibili sperimentalmente, possiamo infine calcolare i numeri cumulativi di nucleoni che completano le ‘shell’ nucleari (indicati in rosso) $2, 8, 20, 34, 58, 92, 132$. Il confronto con i numeri magici osservati $8, 20, 28, 50, 82$ e $126$ non è soddisfacente per cui dobbiamo concludere che \emph{la forma del potenziale deve essere modificata}.
\begin{table*}[h]
	\begin{tabular}{|ccccccc|}
		\hline
		\multicolumn{7}{|c|}{Zeros of Bessel’s Functions of the First Kind} \\ \hline
		\multicolumn{1}{|c|}{Number of Zeros} & \multicolumn{1}{c|}{$J_{0}(x)$} & \multicolumn{1}{c|}{$J_{1}(x)$} & \multicolumn{1}{c|}{$J_{2}(x)$} & \multicolumn{1}{c|}{$J_{3}(x)$} & \multicolumn{1}{c|}{$J_{4}(x)$} & $J_{5}(x)$ \\ \hline
		\multicolumn{1}{|c|}{$1$} & \multicolumn{1}{c|}{$2.40483$} & \multicolumn{1}{c|}{$3.83171$} & \multicolumn{1}{c|}{$5.13562$} & \multicolumn{1}{c|}{$6.38016$} & \multicolumn{1}{c|}{$7.58834$} & $8.77148$ \\ \hline
		\multicolumn{1}{|c|}{$2$} & \multicolumn{1}{c|}{$5.52008$} & \multicolumn{1}{c|}{$7.01559$} & \multicolumn{1}{c|}{$8.41724$} & \multicolumn{1}{c|}{$9.76102$} & \multicolumn{1}{c|}{$11.06471$} & $12.3386$ \\ \hline
		\multicolumn{1}{|c|}{$3$} & \multicolumn{1}{c|}{$8.65373$} & \multicolumn{1}{c|}{$10.17347$} & \multicolumn{1}{c|}{$11.61984$} & \multicolumn{1}{c|}{$13.0152$} & \multicolumn{1}{c|}{$14.37254$} & $15.70017$ \\ \hline
		\multicolumn{1}{|c|}{$4$} & \multicolumn{1}{c|}{$11.79153$} & \multicolumn{1}{c|}{$13.32369$} & \multicolumn{1}{c|}{$14.79595$} & \multicolumn{1}{c|}{$16.22347$} & \multicolumn{1}{c|}{$17.61597$} & $18.98013$ \\ \hline
		\multicolumn{1}{|c|}{$5$} & \multicolumn{1}{c|}{$14.93092$} & \multicolumn{1}{c|}{$16.47063$} & \multicolumn{1}{c|}{$17.95982$} & \multicolumn{1}{c|}{$19.40941$} & \multicolumn{1}{c|}{$20.82693$} & $22.2178$ \\ \hline
	\end{tabular}
\end{table*}
\bigskip

Una prima ovvia modifica non può essere che quella di richiedere che il potenziale abbia una \emph{profondità finita e non infinita}
con una \emph{risalita ripida} ma \emph{non verticale} in $r=R$ così da essere privo di punti assai poco fisici di non derivabilità.
Una espressione semplice che soddisfi questi requisiti è data dal \textbf{potenziale di Saxon-Wood}
\[
V_{SW}(r) = -\frac{V_{0}}{1 + \exp{\left( \frac{r-R}{d} \right)}}
\]
\begin{marginfigure}
	\includegraphics{figs/saxon-woods-potential}
	\caption{Woods–Saxon potential for $A = 50$,with $d = 0.5 \, fm$ and $R=4.6 \, fm$.}
	\label{fig:saxon-woods-graph}
\end{marginfigure}
dove $V_{0}$ è la profondità della buca di potenziale (dell’ordine di $50 \, MeV$), $R=r_{0}A^{1/3} (r_{0} = 1.24 \, fm)$ è il raggio nucleare e $d = 0.52 \, fm$ lo spessore dell’alone nucleare.

I livelli energetici del potenziale sferico
di Saxon-Woods, confrontati con quelli
della buca di potenziale sferica infinita,
sono mostrati in figura~\ref{fig:energetic-levels-sw}. Come
si vede risultano confermati i numeri
magici 2, 8 e 20, quest’ultimo con
maggior nettezza del caso precedente,
mentre i numeri magici più alti sono
errati.
\begin{marginfigure}
	\includegraphics{figs/energetic-levels-sw}
	\caption{Energetic levels confronting based on Saxon-Woods potential compared to the infinite spherical well.}
	\label{fig:energetic-levels-sw}
\end{marginfigure}
In particolare, il numero magico
28 sembra davvero problematico
poiché nessuno dei livelli successivi al
2s porta con sé 8 nucleoni (il livello 1f ne porta ben 14!).
La soluzione del problema discusso fu trovata introducendo nel potenziale nucleare medio un termine spin-orbita,
modellato sul caso del potenziale elettromagnetico dell’elettrone legato al nucleo atomico che discuteremo nel paragrafo seguente.


\subsection{L'interazione spin orbita nell'atomo}\label{sec:spin-orbita-atomo}
Per cominciare, ricordiamo che una \emph{carica elettrica puntiforme} che si muova in modo tale che possa essere trascurato l’irraggiamento di campi elettromagnetici, possiede solo una energia potenziale elettrostatica $U_{em} = qV$. Infatti, dalla relazione generale che lega il potenziale alla forza
\[
dU = - \bm{F} \cdot d \bm{s}
\]
nel caso della forza elettromagnetica otteniamo
\begin{align*}
	dU_{em} &= - \bm{F} \cdot d \bm{s} \\
	& = - (q \bm{E} + q \bm{v} \wedge \bm{B}) \cdot d \bm{s}
	= - q \bm{E} \cdot d \bm{s} - q \bm{v} \wedge \bm{B} \cdot d \bm{s} \\
	& = q \nabla V \cdot d \bm{s} = d(qV)
\end{align*}
da cui
\begin{equation}
	U_{em} = qV
	\label{eq:potential-energy-electrostatic}
\end{equation}
Se invece la carica \emph{non è puntiforme}, ci sono ulteriori gradi di libertà che contribuiscono all’energia che aggiungono al potenziale elettrostatico termini dipendenti dal \emph{momento di dipolo elettrico e dal momento di dipolo magnetico}, legati essenzialmente alla distribuzione spaziale di cariche e correnti.

Dato che sappiamo che \emph{le particelle fondamentali non possiedono momento di dipolo elettrico} ci occuperemo solamente del contributo al potenziale derivante dal momento di dipolo magnetico.
A questo proposito, si può derivare una formula di validità generale analizzando il caso particolare di una distribuzione di correnti particolarmente semplice: quella di una \emph{spira rettangolare percorsa da una corrente costante immersa in un campo elettrico e magnetico uniformi}. E’ facile rendersi conto che una tale distribuzione spaziale di corrente determina l’insorgere di una azione da parte del solo campo magnetico. Si tratta di una azione assente nel caso puntiforme, sostanzialmente dovuta al \emph{termine di Lorentz} della forza elettromagnetica.

A causa della diversa orientazione delle correnti rispetto al campo magnetico, le forze in gioco sui lati lunghi sono compensate dalla rigidità della spira, mentre quelle operanti sui lati corti determinano di fatto una coppia che tende a ruotare la spira attorno all’asse mediano.
\begin{marginfigure}
	\includegraphics{figs/rectangular-coil}
	\caption{Rectangular coil immersed in a uniform magnetic field.}
	\label{fig:rectangular-coil}
\end{marginfigure}
Compreso questo fatto possiamo facilmente calcolare il contributo al potenziale di un elemento $dl$ della spira
\begin{align*}
	d U &= - \bm{F} \cdot d \bm{s} \\
	&= - (id \bm{l} \wedge \bm{B}) \cdot d \bm{s} \\
	& = - (i dl B) ds \cos\left( \frac{\pi}{2} - \theta \right) = -(idlB)\left( - \frac{b}{2} d \theta \right)(\sin \theta) \\
	&= i dl \frac{b}{2} B \sin \theta d \theta
\end{align*}
\begin{marginfigure}
	\includegraphics{figs/rectangular-coil-angle}
	%%\caption{Rectangular coil immersed inside a uniform magnetic field.}%%
	\label{fig:rectangular-coil-angle}
\end{marginfigure}
Integrando l’elemento dl della spira sui due lati corti, ed introducendo il \emph{vettore momento di dipolo magnetico} della spira $\bm{\mu} = i ab \, \hat{\bm{n}}$ dove $\hat{\bm{n}}$ è il versore normale al piano della spira orientato dal verso della corrente secondo la regola della mano destra, otteniamo
\[
dU = i 2a \frac{b}{2} B \sin \theta d \theta = i a bB \sin \theta d \theta =
d(-iabB \cos \theta) = d (-\bm{\mu} \cdot \bm{B})
\]
da cui la seguente espressione del \emph{contributo al potenziale del momento di  dipolo magnetico} $\bm{\mu}$ \emph{immerso nel campo magnetico} $\bm{B}$
\begin{equation}
	U = - \bm{\mu} \cdot \bm{B}
	\label{eq:potential-energy-coil-magnetic-field}
\end{equation}
Il fatto che la formula abbia una struttura del tutto indipendente dalle proprietà geometriche del sistema fisico esaminato, suggerisce che abbia validità del tutto generale.
Sulla base delle (\ref{eq:potential-energy-electrostatic}) e (\ref{eq:potential-energy-coil-magnetic-field}), concludiamo che un \emph{corpo elettricamente carico dotato di momento di dipolo magnetico immerso in un campo elettrico e magnetico ha il seguente potenziale complessivo}
\begin{equation}
	U_{em} = qV - \bm{\mu} \cdot \bm{B}
	\label{eq:total-potential-energy-charged-body}
\end{equation}
Il potenziale (\ref{eq:total-potential-energy-charged-body}) può essere utilizzato nel caso dell’elettrone legato al nucleo atomico. Infatti, da un lato l’elettrone possiede un momento di dipolo magnetico, dall’altro risulta immerso nel campo magnetico generato dal moto apparente del nucleo carico positivamente.
\bigskip

Il fatto che l’elettrone possieda un momento di dipolo magnetico trova la sua evidenza ultima nei fatti sperimentali. Tuttavia, è istruttivo avvicinarci gradualmente a questo concetto assumendo per cominciare la prospettiva della fisica classica.
Pensiamo allora l’elettrone come una \emph{distribuzione volumetrica di carica elettrica in rotazione attorno ad un asse baricentrico}. Tralasciando il problema della forza necessaria per tenere insieme una tale distribuzione di carica, immaginiamo per semplicità che la carica elettrica $–e$ abbia una distribuzione anulare, rotante attorno all’asse normale passante per il centro.
Definendo $S$ il \emph{momento angolare intrinseco} generato dalla rotazione del sistema, è facile trovare la seguente relazione
\[
\bm{\mu} = - \frac{e}{2m} \bm{S}
\]
la quale chiarisce che una \emph{distribuzione estesa di carica elettrica rotante, possiede inevitabilmente un momento di dipolo magnetico}.
Possiamo ora domandarci se tale relazione valga davvero per l’elettrone che sappiamo essere invece una \emph{particella puntiforme, dotata di spin ed in generale soggetta alle leggi della meccanica quantistica e non a quelle della fisica classica}. La risposta è piuttosto sorprendente poiché i fatti sperimentali mostrano che, al netto di alcune importanti correzioni, la formula mantiene una sua validità. Rimane vero infatti che una \emph{particella dotata di carica elettrica e spin possiede inevitabilmente un momento di dipolo magnetico}. Si parla però di \emph{spin quantomeccanico}, dunque di un momento angolare intrinseco con valori quantizzati e stati corrispondenti rispettivamente agli autovalori ed agli autostati degli \emph{operatori dello spin}.

Infine, il fattore numerico non è quello giusto e deve essere moltiplicato da un termine correttivo detto \emph{rapporto giromagnetico}.
Tenendo presente che le componenti cartesiane degli operatori dello spin hanno autovalori multipli di $\hslash$, l’espressione quantomeccanica del \emph{momento di dipolo magnetico} assume la forma seguente
\begin{equation}
	\hat{\bm{\mu}} = - g  \frac{e\hslash}{2 m} \frac{\hat{\bm{S}}}{\hslash}
	\label{eq:quantum-magnetic-dipole-moment}
\end{equation}
dove, $g$ è il rapporto giromagnetico che nel caso delle particelle puntiformi vale $g=2$ (predicibile teoricamente attraverso l’equazione relativistica di Dirac a meno di correzioni elettrodinamiche),$\frac{e\hslash}{2m}$ è il \emph{magnetone di Bohr} (che avevamo introdotto a pag. \pageref{eq:qm-magnetic-dipole-moment}) ed $S$, infine, una generica componente cartesiana dell’operatore dello spin.
Dato che solo gli operatori ${\hat{S}}^{2}$ e ${\hat{S}}_{z}$ possono assumere valori definiti, ne deriva che il momento di dipolo magnetico potrà assumere i seguenti valori
\begin{align*}
	| \bm{\mu} | &= -g  \frac{e\hslash}{2m} \sqrt{ s(s+1) } \qquad s= 0, \frac{1}{2}, \frac{3}{2}, \dots \\
	\mu_{z} &= -g  \frac{e\hslash}{2m} s_{z} \qquad \quad \qquad-s \leq s_{z} \leq +s
\end{align*}
Per l'elettrone che possiede uno \emph{spin} $s = \frac{1}{2}$ otteniamo
\begin{align*}
	| \bm{\mu} | &= -g  \frac{e\hslash}{2m} \sqrt{ \frac{3}{4} } \qquad \qquad \quad  s= 0, \frac{1}{2}, \frac{3}{2}, \dots \\
	\mu_{z} &= -g  \frac{e\hslash}{2m} \left( \pm \frac{1}{2} \right) \qquad \quad -s \leq s_{z} \leq +s
\end{align*}
ovvero due soli possibili valori per la componente definita del momento di dipolo magnetico.

Il nucleo, in quanto dotato di carica elettrica positiva, crea attorno a sé un campo elettrico radiale che trattiene l’elettrone (carico negativamente) su di una orbita chiusa per mezzo della forza coulombiana.
E’ essenziale notare però che un osservatore solidale con l’elettrone vedrebbe, in ogni istante,  un nucleo carico in moto con velocità $- \bm{v}$ opposta a quella $\bm{v}$ dell’elettrone, per cui su di esso deve agire pure il \emph{seguente campo magnetico}\marginnote{$ c^2 = \frac{1}{\mu_0 \epsilon_0}$}
\begin{gather*}
\bm{B} = \frac{\mu_{0}}{4 \pi} \frac{q_{nuc} \bm{v}_{nuc} \wedge \bm{r}}{r^{3}} = \frac{\mu_{0}}{4 \pi} \frac{q_{nuc} (-\bm{v}_{el}) \wedge \bm{r}}{r^{3}} = \frac{\mu_{0}}{4 \pi} \frac{q_{nuc} (\bm{r} \wedge m\bm{v}_{el})}{mr^{3}} \\
= \frac{\mu_{0}}{4 \pi} \frac{q_{nuc}}{m r^{3}} \bm{L}  = \frac{q_{nuc}}{4 \pi \epsilon_{0}r^{2}} \frac{1}{mc^{2}r} \bm{L} = \frac{1}{mc^{2}}  \frac{|\bm{E}|}{r} \bm{L}
\end{gather*}
da cui
\begin{equation}
	\bm{B} = \frac{1}{mc^{2}} \frac{1}{r} \bigg|\frac{ \partial U }{ \partial r }  \bigg| \bm{L}
	\label{eq:magnetic-field-on-atomic-electron}
\end{equation}
Ora abbiamo tutti gli elementi per giungere al risultato finale.
Infatti il campo magnetico (\ref{eq:magnetic-field-on-atomic-electron}) agente sull’elettrone si accoppia al suo momento di dipolo magnetico intrinseco (\ref{eq:quantum-magnetic-dipole-moment}) determinando il seguente contributo (\ref{eq:potential-energy-coil-magnetic-field}) al \emph{potenziale elettromagnetico}
\begin{equation}
	U = \frac{g \left( \frac{e\hslash}{2m} \right)}{m c^{2}} \frac{1}{r} \bigg| \frac{ \partial U }{ \partial r }  \bigg| \bm{L} \cdot \bm{S}
	\label{eq:spin-orbit-em-potential}
\end{equation}
spesso indicato come \textbf{potenziale eletromagnetico spin-orbita} proprio perchè determinato dall’accoppiamento tra i i momenti angolari orbitale ed intrinseco dell’elettrone.
Richiamando la (\ref{eq:total-potential-energy-charged-body}) otteniamo infine la seguente espressione del \emph{potenziale dell’elettrone atomico}
\begin{equation}
	U = qV + \frac{g \left( \frac{e\hslash}{2m} \right)}{m c^{2}} \frac{1}{r} \bigg| \frac{ \partial U }{ \partial r }  \bigg| \bm{L} \cdot \bm{S}
	\label{eq:total-em-potential-spin-orbit}
\end{equation}
Il fatto che il potenziale dell’elettrone dipenda dal \emph{prodotto scalare tra momento angolare orbitale e spin} è degno di nota. Infatti, ciò significa che a seconda dell’angolo formato da questi due vettori la ‘posizione’ del livello energetico varia all’interno di un intervallo centrato sul valore $qV$
\[
U = qV \pm \frac{g \left( \frac{e\hslash}{2m} \right)}{m c^{2}} \frac{1}{r} \bigg| \frac{ \partial U }{ \partial r }  \bigg| \bigg|\bm{L}  \bigg|  \bigg|\bm{S} \bigg|
\]
A causa di questo fatto, la ‘luce’ emessa o assorbita dall’elettrone nel corso di un processo di diseccitazione o eccitazione dovrebbe possedere una gamma continua di frequenze centrate attorno ad un valore centrale calcolabili con la formula $\nu = E / h$.
In realtà i dati sperimentali mostrano chiaramente che una tale distribuzione continua di frequenze non si realizza. Piuttosto queste si dividono in coppie di righe ben definite assai ravvicinate (la cosiddetta struttura fina).

Tale fatto, come vedremo nel prosieguo, può essere spiegato solo sostituendo le quantità vettoriali della espressione classica con i corrispondenti operatori quantomeccanici. Allora accadrà che il «prodotto scalare» tra gli operatori del momento angolare orbitale e dello spin fornirà non una distribuzione continua di valori, ma una distribuzione discreta che, nel caso dello spin $s=1/2$, si ridurrà a due soli valori capaci di interpretare perfettamente i dati sperimentali.

\subsection{L'interazione spin-orbita nel nucleo}\label{sec:spin-orbit-interaction-nucleus}

Sulla falsariga del potenziale elettromagnetico (\ref{eq:total-em-potential-spin-orbit}), Mayer, Haxel, Suess e Jensen (il primo e l’ultimo ricevettero per questo contributo il premio Nobel nel 1963) ipotizzarono - su suggerimento di E. Fermi - che anche le interazioni forti tra nucleoni fossero caratterizzate da una interazione spin orbita in virtù di una supposta analogia strutturale tra interazione elettromagnetica ed interazione forte dotata anch’essa di un termine tipo Lorenz dipendente dalla velocità.
\bigskip

Premesso il passaggio agli operatori quantomeccanici, l’assunzione di tale ipotesi comportava che, accanto al termine centrale di Saxon-Woods, l’operatore potenziale medio nucleare della interazione forte tra nucleoni dovesse contenere pure un secondo termine centrale del tipo spin-orbita acquisendo così la forma seguente
\begin{equation}
	\boxed{\hat{V}_{forte} = V_{SW}(r) + V_{ls}(r) \hat{L} \cdot  \hat{S}}
	\label{eq:mean-nuclear-potential-operator}
\end{equation}

Ottenuto questo risultato, il passo successivo non può che essere quello di calcolare i valori di aspettazione di tale potenziale che dovranno essere poi confrontati con quelli sperimentali.

Come noto, secondo le regole della meccanica quantistica, il valore di una variabile dinamica $o$ in uno stato quantomeccanico descritto dalla funzione d’onda $\psi$ è dato da un certo integrale convolutivo della funzione d’onda $\psi$ e dell’operatore $O$ associato alla variabile dinamica detto \emph{valore di aspettazione dell’operatore} espresso dall'equazione (\ref{eq:observable-qm-axiom}).
Sostituendo la (\ref{eq:mean-nuclear-potential-operator}), e tenendo conto che lo stato quantomeccanico dei nucleoni è descritto da funzioni delle variabili radiali ed angolari dipendenti dai numeri quantici del momento angolare $l$ ed $m$ (vedi la (\ref{eq:spatial-part-wave-function}) ), otteniamo
\begin{equation}
	\begin{aligned}
		\langle \hat{V}_{forte} \rangle &= \int_{V} \overline{\varphi}_{l,m}(r,\theta,\phi) [V_{SW}(r)+V_{ls}(r) \hat{L} \cdot \hat{S}]\varphi_{l,m}(r,\theta,\phi) \ dV  \\
		&= \int_{V} \overline{\varphi}_{l,m} V_{SW}(r)\varphi_{l,m} \ dV + \int_{V}\overline{\varphi}_{l,m}(\bm{r},t) V_{ls}(r)\, \hat{L} \cdot \hat{S} \, \varphi_{l,m} \ dV
	\end{aligned}
	\label{eq:mean-potential-calc}
\end{equation}
Per procedere nel calcolo consideriamo l’ultimo integrale.
L’azione dell’operatore $\hat{L} \cdot  \hat{S}$ sullo stato quantomeccanico $\varphi_{l,m}$ può essere determinata introducendo l’operatore \emph{momento angolare totale} $\hat{J}$ del nucleone, dato dalla somma degli operatori \emph{momento angolare orbitale} $\hat{L}$ e \emph{momento angolare di spin} $\hat{S}$
\begin{equation}
	\hat{J} = \hat{L} + \hat{S}
	\label{eq:total-angular-momentum-operator}
\end{equation}
Infatti, calcolando il ‘quadrato’ dell’operatore $\hat{J}$ si ottiene
\[
	\hat{J}^{2} = \hat{L}^{2} + \hat{S}^{2} + 2 \hat{L} \cdot \hat{S}
\]
da cui si ricava la seguente relazione operatoriale
\[
	\hat{L} \cdot \hat{S} = \frac{1}{2} (\hat{J}^{2} - \hat{L}^{2} - \hat{S}^{2})
\]
Applicando tale espressione al generico stato quantomeccanico del nucleone nel nucleo otteniamo la seguente espressione
\[
	\hat{L} \cdot \hat{S} \varphi_{l,m} = \frac{1}{2} (\hat{J}^{2} - \hat{L}^{2} - \hat{S}^{2}) \varphi_{l,m} = \frac{\hslash^{2}}{2} [j(j+1) - l(l+1) -s(s+1) ]\varphi_{l,m}
\]
che sostituita nella (\ref{eq:mean-potential-calc}) fornisce
\begin{align*}
	\langle \hat{V}_{forte} \rangle & = \int_{V} \overline{\varphi}_{l,m} V_{SW}(r)\varphi_{l,m} \ dV + \\
	&+\frac{\hslash^{2}}{2} [j(j+1) - l(l+1) -s(s+1) ]
	\int_{V}\overline{\varphi}_{l,m}(\bm{r},t) V_{ls}(r)\,  \varphi_{l,m} \ dV
\end{align*}
Ora si noti che le espressioni integrali, note le funzioni della coordinata radiale che descrivono i potenziali di Saxon-Woods e spin-orbita, assumeranno un definito valore numerico per cui potremo scrivere
\begin{equation}
	\langle \hat{V}_{forte} \rangle = \beta' + \beta\frac{\hslash^{2}}{2} [j(j+1) - l(l+1) -s(s+1) ]
	\label{eq:mean-potential-calc2}
\end{equation}
Richiamando ora il teorema di somma dei momenti angolari, dalla (\ref{eq:total-angular-momentum-operator}) otteniamo i seguenti \emph{numeri quantici del momento angolare totale}
\[
	j = |l-s|, \dots , |l+s| \qquad l = 0,1,2,\dots \quad s = 0, \frac{1}{2}, \frac{3}{2}, \dots
\]
che, nel caso dei nucleoni che hanno spin $s=1/2$, forniscono
\[
	j = l - \frac{1}{2}, l + \frac{1}{2} \qquad l = 0,1,2,\dots
\]
Sostituendo infine nella (\ref{eq:mean-potential-calc2}) otteniamo facilmente i \emph{valori di aspettazione del potenziale medio nucleare dei nucleoni nel nucleo}
\begin{equation}
	\langle \hat{V}_{forte} \rangle = \beta' + \beta\frac{\hslash^{2}}{2}
	\begin{cases}
		\qquad l \qquad \quad j = l+\frac{1}{2} \\ \\
		-(l+1) \qquad j = l -\frac{1}{2}
	\end{cases}
	\quad l = 1,2,\dots
	\label{eq:expected-value-shell-model-potential}
\end{equation}
Si noti che ciascun livello energetico $\beta'$ del potenziale di Saxon-Woods - ad eccezione dei livelli corrispondenti ad $l=0$ per quali si ha $\hat{L} \cdot \hat{S} \varphi_{l,m} = 0$ (vedi la (\ref{eq:mean-potential-calc})) - risulta «splittato» in due sottolivelli spaziati in misura crescente con il momento angolare orbitale $l$ del nucleone
\begin{equation}
	\langle \Delta \hat{V}_{forte} \rangle = \frac{\beta \hslash^{2}}{2}(2l+1)
	\label{eq:saxon-woods-potential-energy-levels-splitting}
\end{equation}
\begin{figure}
	\centering
	\includegraphics{figs/shell-model-energy-level-final}
	\caption{Livelli energetici derivanti dal modello a shell a confronto con approssimazioni precedenti.}
	\label{fig:shell-model-energy-level-final}
\end{figure}
E’ importante precisare – come mostreremo tra poco - che i dati sperimentali richiedono un coefficiente $\beta$ negativo il che significa che i sottolivelli $j=l-1/2$ hanno una energia superiore a quelli $j=l+1/2$.
Fatte queste premesse lo schema dei livelli del potenziale nucleare medio, completo del contributo spin-orbita, è mostrato in figura~\ref{fig:shell-model-energy-level-final}.
Come anticipato, i livelli energetici ns ($l=0$) non vengono separati, al contrario dei livelli $p, d, f$ etc \ldots
che vengono divisi con una separazione sempre più ampia.

Per calcolare i numeri magici associati a tale potenziale è necessario precisare in quale modo i nucleoni andranno a disporsi nei diversi sottolivelli, tenendo presente che nello stato del nucleo di minima energia i livelli saranno occupati a partire da quelli di energia inferiore. Dato che, in accordo con il principio di esclusione di Pauli, su ciascuno stato quantico potrà risiedere un solo nucleone (si ricordi che, a differenza del modello a gas di fermioni, lo spin interviene esplicitamente nella definizione dello stato quantico del nucleone) ciò significa che uno dopo l’altro i nucleoni andranno a riempire i livelli energetici partendo dal basso.

Ora, il generico livello con numero quantico del \emph{momento angolare totale} $j$ (nella notazione atomica indicato in basso a destra) ha degenerazione $2j+1$, ovvero contiene $2j+1$ stati quantomeccanici identificati dal valore del numero quantico magnetico $m=-j, -j+1,\dots,j-1, j$.
Ciò significa che il sottolivello $j=l+1/2$ avrà degenerazione $2l+2$ e quello $j=l-1/2$ degenerazione $2l$.
Dunque, i 6 nucleoni che possono ad esempio stabilirsi sui livelli np ($l=1$) del potenziale di Saxon-Woods si suddivideranno in $2l+2=4$ nucleoni sul sottolivello $j=3/2$ e $2l=2$ nucleoni sul sottolivello $j=1/2$.

Fatte queste premesse possiamo comprendere nei dettagli in Figura~\ref{fig:shell-model-energy-level-final}.


Tra i vari dettagli, vale la pena notare come lo «splitting» del livello $1f$ (unitamente al valore negativo di $\beta$) stacchi 8 dei 14 nucleoni nello stato $1f_{\frac{7}{2}}$ spiegando l’origine del problematico numero magico 28.
Anche nel caso dei numeri magici più elevati lo «splitting» dei livelli risulta determinante nella loro strutturazione a bande in perfetto accordo con i dati sperimentali.

Per finire, sottolineiamo che la struttura dei livelli mostrata vale per tutti i nucleoni e dunque sia per i neutroni che per i protoni con l’unica differenza che, a causa della repulsione coulombiana, i livelli energetici dei protoni dovranno essere spostati un po' più verso l’alto.

\subsection{Previsioni del modello a Shell}\label{sec:forecast-shell-model}

Il modello a shell introduce un potenziale nucleare medio i cui livelli energetici si accordano in modo soddisfacente con i dati sperimentali ed è in grado di fare numerose previsioni.
Come già osservato, nello stato fondamentale del nucleo, gli $N$ neutroni ed i $Z$ protoni occuperanno i rispettivi livelli energetici a partire dal basso. Nel fare questo, completeranno un certo numero di livelli ma vi sarà in generale anche un resto che andrà a disporsi sul livello energetico più alto. I neutroni o protoni che completano i livelli sono detti neutroni o protoni del ‘core’, quelli invece che occupano in modo incompleto il livello di energia più elevata sono detti neutroni o protoni di valenza.
Il modello a shell, nella sua forma più semplice (a singola particella) assume che
\begin{itemize}
	\item i neutroni e protoni del ‘core’ non contribuiscano alle proprietà del nucleo;
	\item i neutroni e protoni di valenza che possono appaiarsi non contribuiscano alle prorpietà del nucleo;
    \item solo i singoli neutroni e protoni di valenza non appaiati contribuiscano alle proprietà del nucleo.
\end{itemize}
Premesso che la vera giustificazione di tali ipotesi risiede nella loro capacità di prevedere correttamente le proprietà dei nuclei, ci si può chiedere se vi sia un fondamento teorico alla idea che i nucleoni del ‘core’ non diano alcun contributo.
\begin{marginfigure}
	\includegraphics{figs/magic-numbers-shell-model}
	%    \caption{This is a margin figure.}
	\label{fig:magic-numbers-shell-model}
\end{marginfigure}
Consideriamo il caso del momento
angolare nucleare. Tali ipotesi implicano
che lo spin del nucleo sia determinato dai
soli neutroni o protoni di valenza spaiati,
per cui la somma dei momenti angolari
totali dei neutroni e protoni del ‘core’, e
dunque di ciascuna ‘shell’ completa,
debba annullarsi.
Ora, se una ‘shell’ di determinato
momento angolare orbitale $j$ è
completa, saranno occupati anche tutti
gli stati $(j,-j), (j,-j+1)...(j,j-1), (j,j)$ che così
concorrono con uguale peso a formare lo
stato quantistico complessivo (infatti, una
ipotetica misura di momento angolare,
deve trovare con uguale probabilità i
suddetti valori). Ma l’unico stato quantico
che, fissato il momento angolare $j$, da
uguale peso a tutti i diversi valori di $j_{z}$ è lo
stato di momento angolare nullo, per cui
deduciamo che i neutroni o protoni di una
‘shell’ completa non contribuiscono al
momento angolare totale del nucleo.
Sulla base di queste premesse possiamo fare alcune semplici previsioni.
Ad esempio ne deriva che, secondo il modello a ‘shell’, non solo i nuclei doppiamente magici (che giocano un ruolo simile a quello dei gas nobili nella fisica atomica) ma tutti i nuclei \emph{pari-pari} \emph{non possiedono spin}, un fatto pienamente confermato dai dati sperimentali.
Consideriamo ora il caso dei nuclei \emph{pari-dispari}.
Nel caso dell’ossigeno (magico in $Z=8$) prevediamo per la serie isotopica i seguenti spin: $\isotope[14][8]{O}: s= 0,
\isotope[15][8]{O}: s= 1/2, \isotope[16][8]{O}: s= 0, \isotope[17][8]{O}: s= 5/2, \isotope[18][8]{O}: s= 0$, e così via a valori
alternati $0, 5/2$ fino all’isotopo $\isotope[23][8]{O}: s= 1/2$.
Anche in questo caso i dati sperimentali confermano la previsione.
Non sempre il modello a shell a singola particella coglie nel segno. Ad
esempio il (\isotope[47][22]{Ti}) dovrebbe avere uno spin $s=7/2$ mentre sperimentalmente
si trova $s=5/2$. In questi casi si ammette che lo spin venga deteminato non
dal solo nucleone di valenza spaiato ma da tutti i nucleoni di valenza
opportunamente accoppiati.
Consideriamo infine il caso dei nuclei \emph{dispari-dispari}. Ad esempio, il nucleo
($\isotope[6][3]{Li}$) ha un protone ed un neutrone spaiati portatori di momento angolare
$J=3/2$ ciascuno che possono comporsi nei valori di spin totale $s=3,2,1,0$.
Nella sua forma basilare, il modello non può dire di più ma non è smentito dal valore sperimentale $s=1$.
\bigskip

Concludiamo questa parte ricordando che il modello a shell può fare anche previsioni sui momenti di dipolo magnetico ed elettrico dei nuclei.
Si tratta però di previsioni problematiche soprattutto nel caso dei momenti di dipolo elettrico che risultano essere nel caso $150<A<190$ e $A>220$ fortemente sottostimati rispetto ai valori sperimentali.
Sono queste le ragioni che hanno spinto i fisici nucleari a sviluppare il \emph{Modello Collettivo} basato su di un potenziale non sfericamente
simmetrico che introduce forti deformazioni nei nuclei pesanti e con esse consistenti valori del momento di dipolo elettrico.


















































