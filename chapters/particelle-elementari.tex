\section{Profilo storico}\label{sec:profilo-storico}

La fisica delle particelle elementari è un’area della ricerca in fisica che si occupa dello studio dei costituenti ultimi della materia e delle loro interazioni (per maggiori dettagli vedi materiale facoltativo – profilo storico).
La scoperta dell’elettrone (J.J Thompson,1897), del protone (E. Rutherford, 1917) e del neutrone (J. Chadwick, 1932) quali componenti fondamentali dell’atomo e dunque della materia ordinaria diedero avvio a questa disciplina.
L’evidenza indiretta del neutrino nel decadimento beta del neutrone (Pauli, 1930) e le successive scoperte del positrone (C. Anderson, 1932) e del muone (C. Anderson, S. Neddermeyer, 1936) nei raggi cosmici, chiarirono che esistevano particelle estranee alla architettura dell’atomo.
La presa di coscienza da parte della comunità che il mondo delle particelle era ben più esteso di quello della materia ordinaria diede avvio alla fisica delle particelle elementari come disciplina autonoma nel senso odierno del termine.
A partire dall’immediato dopoguerra, lo studio delle collisioni tra particelle all’interno di acceleratori sempre più perfezionati e di energia sempre più elevata ha condotto ad un progresso travolgente della disciplina.
Furono trovati un numero relativamente ridotto di nuovi leptoni (da ‘leptos’=leggero, particelle non soggette alla interazione forte) ed un numero elevatissimo di nuove particelle fortemente interagenti dette adroni (da ‘adros’=forte) che venivano scoperte con ritmo inquietante (fenomeno della proliferazione degli adroni).
La strategia sperimentale è stata guidata dal principio che l’energia della collisione tra particelle sia dissipata – attraverso le interazioni in gioco - nella creazione di nuove particelle ragion per cui l’indagine si è orientata verso collisioni di energia sempre più elevata presso acceleratori sempre più grandi dove il limite è stato essenzialmente imposto dalla tecnologia e dai finanziamenti a disposizione.
Sin dall’inizio si comprese che le condizioni di densità di energia estreme create artificialmente all’interno degli acceleratori su spazi e tempi microscopici dovevano essere le condizioni ordinarie dell’universo nei suoi primi istanti di vita, per cui la fisica delle particelle alle alte energie apriva di fatto una finestra sulla fisica fondamentale del cosmo, complementare a quella della tradizionale cosmologia osservativa.
Attraverso le conoscenze accumulate presso le macchine acceleratrici è stato così possibile ricostruire

\section{Aspetti generali del modello standard}\label{sec:aspetti-generali-del-modello-standard}

\section{L'intensità relativa delle interazioni}\label{sec:intensita-relativa-delle-interazioni}

\section{I campi quantizzati}\label{sec:campi-quantizzati}

\section{Simmetria in fisica delle particelle}\label{sec:simmetria-in-fisica-delle-particelle}

